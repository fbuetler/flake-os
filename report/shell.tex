\section{Shell}

The shell is split into two modules, a serial server and the shell itself.
The idea is that the shell should only require an interface to read/write and be completely decoupled form any hardware.
That way, any IO device such as a network socket could be used without major modifications of the shell.

\subsection{Shell}
The core of the shell is a read, evaluate, execute loop.
It repeatedly queries the serial server to get the next character and stores them in a list.
Once a new line is encountered, it extracts the first whitespace separated substring and treats it as a command.
The rest of the input is used as argument to the function associated with the command.
The only exception is the \verb|time| command, which measures the time another command takes.
If this command is encountered, the shell starts a timer, removes \verb|time| from the string, and continues the normal command parsing.
After the call to the command finishes, the timer stops and the difference is printed.

Commands are stored in an array:
\begin{lstlisting}
char *builtin_str[] = {
    "help",
    "exit",
    "echo",
}
\end{lstlisting}

The corresponding function must have the following following prototype: \verb|void func(char *arg);| and its function pointer
is also stored in an array:

\begin{lstlisting}
void (*builtin_func[]) (char *) = {
    &help,
    &shell_exit,
    &echo,
};
\end{lstlisting}

With this setup commands are matched in a simple loop:

\begin{lstlisting}
for (int i = 0; i < num_builtins(); i++) {
    if (strcmp(command, builtin_str[i]) == 0) {
        builtin_func[i](strtok(NULL, ""));
        command_exists = true;
    }
}
\end{lstlisting}

This system was taken from a blog post by Stephan Brennan \footnote{\url{https://brennan.io/2015/01/16/write-a-shell-in-c/}}.

The state of the shell is stored in a struct and initialized on startup:

\begin{lstlisting}
#define RECV_BUFFER_SIZE 64

struct shell_state {
    bool exit; // flag to check if the shell should exit itself
    char line_buffer[RECV_BUFFER_SIZE];
    size_t buffer_count;
    struct aos_rpc *serial_rpc;
    struct aos_rpc *init_rpc;
} shell_state;
\end{lstlisting}

\subsubsection{Improvements}
There are several possible improvements to the current implementation of the shell.

By changing the command function prototype to \verb|errval_t func(char *arg);|, it is possible to store the last return/error code.
As we do not have a \emph{shell programming language} which can do work conditional on the return code, we did not implement this feature.

Instead of pooling new characters from the serial server, an alternative would be to register a callback function from the shell to the serial server.
This way, the interrupts get propagated to all the subscribers.


\subsection{Serial Server}
The serial server can be seen as a rudimentary TTY. It reads and writes through UART and buffers incoming characters.
With only little modification multiple TTYs could be launched, each using a different UART port.

