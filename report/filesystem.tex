\chapter{Filesystem}

\section{Introduction}
For the filesystem, we support SD-cards running FAT32. The filesystem we've implemented 
is built as a service, exposing RPC calls for file operations like opening, closing, writing, reading, etc. 
to user processes, which then can be called by the usual libc functions like \texttt{fopen}, \texttt{fclose}, \texttt{fwrite}.

The alternative, which is building the filesystem as a library, has not be chosen because the filesystem as a service 
gives in our eyes a better abstraction to the underlying FAT32 operations. Another reason why we chose the filesystem as a service
is because it does decrease RPC communication, leading to overall better performance.  

\subsection{Operations}
The operations implemented as a service over RPC calls are:
\begin{itemize}
    \item \texttt{aos\_rpc\_fs\_open}: Open a file with given flags
    \item \texttt{aos\_rpc\_fs\_close}: Close a file
    \item \texttt{aos\_rpc\_fs\_create}: Create a file
    \item \texttt{aos\_rpc\_fs\_read}: Read open file contents into buffer
    \item \texttt{aos\_rpc\_fs\_write}: Write buffer contents to a open file
    \item \texttt{aos\_rpc\_fs\_lseek}: Seeking in a file
    \item \texttt{aos\_rpc\_fs\_dir\_action}: Used to perform the two operations \texttt{mkdir}, \texttt{rmdir}
    \item \texttt{aos\_rpc\_fs\_readdir}: Read directory contents
    \item \texttt{aos\_rpc\_fs\_fstat}: Get file status
\end{itemize}

\section{Block Driver}
After flashing a new filesystem onto the SD-card, we can communicate with it over 
the block driver over the process \texttt{fs}, which has access to the devframe necessary
for using the driver. Since it's the only process with access to the SD-card, we don't have to worry 
about concurrent read/writes in the filesystem.

\subsection{Performance Improvements}
To improve the block driver, we note that we always only read/write blocks of the 
same size with it. Thus, it's enough to only set the block-size once in the beginning,
instead of setting it every time we read/write a block. As we will see during the 
performance analysis in the next section, this will already give a 2x speedup.

There are many other performance improvements which could be made.
Very interestingly to me with respect to FAT32 is that reading/writing whole 
clusters at once could be done much faster. As can be seen in the next section,
sending commands over the block driver alone will take a lot of time, 
so reducing the number of commands sent to the block driver can be of essential 
importance for performance.

It's also possible to read/write specific block-lengths with each write.
However, this change in block-lengths needs to be communicated to the device,
thus another command needs to be send to the SD-reader. We immediately see 
that this constitutes a tradeoff between send-commands and DMA-access time.
While always reading large blocks has few send-commands, DMA over large memory 
takes longer than for small blocks. It could have been interesting to also 
check if or at which point it makes sense to work with variable-sized read/write 
commands.

As a last possibility for improving performance, it could have been possible 
to fetch information from the SD-card to obtain more information of frequencies 
which could be used to allow for faster access speed.

\subsection{Performance Analyis}

\section{FAT32 Implementation}

\subsection{Assumptions}
We assume we receive a valid FAT32 filesystem. We thus don't do any additional 
checking/fixing/infering of the filesystem present.

\subsection{Abstractions}
Wherever possible, we tried to abstract away sectors as much as possible.
Instead, we focused on clusters and indeces into clusters, so that 
we wouldn't have to worry too much about cluster traversals in terms of 
sectors, where in which cluster we are, etc. Seeing FAT32 as a linked-list 
of clusters is much easier than seeing it as a linked-list of sector-arrays.

For buffers which we use to read/write to, we introduced 
\begin{minted}{c}
struct phys_virt_addr {
    // physical address of memory segment
    lpaddr_t phys;
    // virtual address of memory segment
    void *virt;
    // last sector read/written
    uint32_t last_sector;
    // is the buffer dirty?
    bool dirty;
};
\end{minted}
we use it to keep track of where we read/write to. The last two members 
are important for caching, which will be explained laster.




\subsection{Caching}
One file write/read can potentially look at one FAT-sector multiple times,
or could write to the same data sector multiple times. Since every 
read/write from/to a sector on the SD-card is very expensive, a big 
performance increase was the idea of caching reads/writes to the disk.

In a first attempt, we never performed a read from the SD-card if 
the last read/write to/from the SD-card was to the same sector. This can 
be pretty easily by just remembering the last sector each time.

A next observation was that there is much locality in a directory-entry 
as well as a FAT-sector, since many requests looked at closed-by indeces
of a sector. For that reason, we decided to use a different buffer 
for storing any read/writes to a data-sector and a different one for 
FAT sectors (and directory-entry-sectors). 
A write would still always write, it is write-through.

Using these two different buffer, we benefit from locality.
To extend this idea even further, an LRU-approach with $N$ different 
buffer would lead to even fewer reads from the SD-card.
TODO

\subsection{Limitations}
Due to time-constraints, no support for timestamps or read-only files is implemented.
Also, no estimates of free clusters are used.
\section{Filesystem Implementation}
To implement the filesystem, we tried to keep the original approach of \texttt{ramfs}
as much as possible and replaced functions like \texttt{ramfs\_open}, \texttt{ramfs\_close}
have been replaced by \texttt{aos\_rpc\_fs\_open}, \texttt{aos\_rpc\_fs\_close}.

We store a opened file in a simliar struct as in \texttt{ramfs}:
\begin{minted}{c}
struct fat32fs_dirent {
    char *name; 
    size_t size; 
    bool is_dir; 
    // cluster containing the directory-entry
    uint32_t dir_cluster; 
    // index into that cluster
    uint32_t dir_index; 
    // first data cluster of file
    uint32_t start_data_cluster; 
};

struct fat32fs_handle {
    int flags;
    // unique id of the opened file
    fileref_id_t fid;
    domainid_t pid;
    char *path;
    struct fat32fs_dirent *dirent;
    union {
        uint32_t dir_offset;
        uint32_t file_offset;
    } u;
    // current data cluster as 
    // specified by dir_offset/file_offset
    uint32_t curr_data_cluster;
};
\end{minted}
Of particular importance is the member \texttt{fid}.
This ID is a unique ID under all the opened files in the filesystem 
which is used to identify an opened handle. Any RPC request to an opened
file identifies the handle uniquely by this FID.

In the \texttt{fs} process, the following happens when a new file/dir is opened:
\begin{itemize}
    \item Create new handle, store file-info into it.
    \item Assign new unique FID to handle
    \item Add into hashmap: FID $\Rightarrow$ Handle 
    \item Append to hashmap: Path $\Rightarrow$ Handle
\end{itemize}
The first of the two hashmaps is used to find the handle of an opened file by FID in constant 
time. The other stores all FID's which have the same file opened. Using this second hash map,
we resolve many thread-safety issues, which will be explained later.

\subsection{Service}

\subsection{Thread Safety}

\subsubsection{Locking}

\subsubsection{Updating}

\subsection{Security}
For a full-blown OS, it makes sense to have a security layer in place which 
doesn't allow for one process to read/write using FID's of another process.
This could have been done in multiple ways. One could be to authenticate the RPC channels,
so that every FID is also linked to a specific channel. That way, requests to foreign 
FID's could be rejected. This could also have been done with special capabilities to
any opened file/directory. 

As this feature was of rather low priority given the problems we had mentioned
during the integration, no additional security feature was implemented.

\section{ELF Loading}
Due to the problems during integration-phase, we didn't have time to implement
the ELF loading feature.

However, for the sake of completeness, we now state how this would be achieved.
Given our existing code, the approach we would follow is to use 
the file system to load the complete file into memory. After that,
any arguments to the program would be passed to by the shell.
To actually load the in-memory program, we can use existing functionality 
we have in our \texttt{spawn} library, where multiboot programs 
are also loaded into memory. After this step, we would start the program 
in the same way as we did for any other program.

\section{Performance Analysis}

\section{Issues}