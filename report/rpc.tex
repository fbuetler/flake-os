\chapter{RPC}

We now briefly mention how we combined our LMP and UMP frameworks into a generic message passing framework, which
is Remote Procedure Call (RPC).

Essentially, using objects containing a union of an LMP channel and a UMP channel as well as a flag designating what kind of
channel this actually is (UMP or LMP), we were able to wrap around functions of LMP and UMP. Since UMP and LMP operate under 
very similar assumptions and invariants, it was an easy task to implement generic functions which work for both sides.

However, we noticed that sometimes, one just has to resolve to working with the concrete functions (i.e. concrete LMP functions).
This is because handler-functions of LMP and UMP are different. They work and are listened to differently.
We were not able to abstract it away completely, nor did we see a reason to.

This unification required a lot of refactoring, which was a very tedious task. Initially, when implementing a framework
based on LMP, we didn't know that UMP of a later milestone should also be working under the same abstraction,
so our RPC implementation was basically synonymous with our LMP implementation before that.

Please note that we changed the signature of \sytx{aos_rpc_init} due to this unification. Due to the vast differences
in arguments for an LMP init function and a UMP init function we did not find a good way to unify these into one function.
There is a way with a tagged union of argument structs, but we do not consider this to be good design. Since a caller always
knows what kind of channel she is creating, such an abstraction is unnecessary. Thus, we refer in the comments to the
LMP and UMP init functions. In order to implement the grading API we decided for \sytx{aos_rpc_init} to simply wrap
\sytx{aos_lmp_init}.