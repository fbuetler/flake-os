\chapter{Simple shell and Power Management}

\section{Ouput}

When someone want to print a character, we wait until the graphical device is
ready and then we send it to make it appear on the screen.

\section{Input}

Input works in a similar fashion. We wait until the device says there is data to
be read and then we read it into local memory.  I did first a mistake by
confusing 0x4 with 1 << 4 and hence the 3rd bit was checked instead of the
fourth. This lead to make the program progress even if there is no character,
resulting in bad performance.

\section{Simple shell}

Implementing a simple shell requires to stores every character that is entered
until the return key is pressed. This is done in a char array. The char array
allow a max input of 255 characters as a first choice.  Once the return key is
entered, the string (stored in the char array) is terminated and evaluated. If
it matches an implemented command, it executes it.  As printf uses the function
we implement in **Input**, we can use it to print stuff on the console. We
provide a 'hello' and 'help' command that just print some text back to the user,
otherwise the input is ignored.  I also implemented the functionality of ctrl+c
to abort an input. Moreover, there is the functionality of backspace implemented
as well. This is done by moving the curse one character back, printing a space
i.e. an empty character and then moving the character once back again.  To make
things a bit more aestetic, a '\$' is printed to start an input and a newline is
started when an input is evaluated.

\section{Shutdown/Reboot}

Shutdown and reboot is all about reading the specification. I decided to use the
SMC32 calling convention and a fast call. First I had trouble understanding what
the value of the service call range was. But it turned out its the "Owning
entity number" i.e. 4. I also had a bug that confused me for a while but it
turned out, I was just so focused on getting the value right that I forgot to
call the "smc" assembly function, therefore the value in the register was never
evaluated.  The difference of shutdown and reboot is just an offset of 1 of the
value stored in the register.  Most challening part of this task was to
understand the specification correctly and getting the assembly instructions
right.
