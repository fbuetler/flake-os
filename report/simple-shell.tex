\chapter{Simple Shell and Power Management}
For the first milestone, we want to briefly outline the problems we faced,
the solutions we found, and the approached we followed. 
While there is not much to tell, we still want to provide a short account of this introduction into the project.

Of course, note first that we use QEMU in this milestone and it's provided UART interface.
In the specification online, we found everything necessary for a successful communication with the interface.

\section{Output}
When wanting to print a character to the screen, we wait until the graphical device is
ready and then we send it to make it appear on the screen.

\section{Input}
Input works in a similar fashion. We wait until the device says there is data to
be read and then we read it into local memory.  On the first try, we made a mistake by
confusing \mintinline{c}{0x4} with \mintinline{c}{1 << 4} and hence the third bit was checked instead of the
fourth. This let the program make progress even if there was no character,
resulting in bad performance.

\section{Simple Shell}
Implementing a simple shell requires to store every character that is entered
until the return key is pressed. This is done in a char array. The char array
allows a maximal input length of 255 characters as a first choice.  Once the return key is
entered, the string (stored in the char array) is terminated and evaluated. If
it matches an implemented command, it executes it.  As \texttt{printf} uses the function
we implemented for output, we were able to use it quite simply to print whole strings to the screen. 
We provide a \texttt{hello} and \texttt{help} command that just print some text back to the user,
otherwise the input is ignored. We also implemented the functionality of \texttt{Ctrl+C}
to abort an input. Moreover, there is the functionality of a backspace implemented
as well. This is done by moving the cursor one character back, printing a space, 
i.e. an empty character and then moving the cursor back again. To make
things a bit more aesthetic, a \texttt{\$} is printed to start an input and a newline is
started when an input is evaluated.

\section{Shutdown/Reboot}
Shutdown and reboot is all about reading the specification. We decided to use the
SMC32 calling convention and a fast call. First, we had trouble understanding what
the value of the service call range was. But it turned out it is the "Owning
entity number", i.e. 4. We also had a bug that was confusing for a while, but it
turned out, we were just so focused on getting the value right that we forgot to
call the \mintinline{asm}{smc} assembly instruction, therefore the value in the register was never
evaluated. The difference of shutdown and reboot is just an offset of 1 of the
value stored in the register. The most challenging part of this task was to
understand the specification correctly and getting the assembly instructions
right.
