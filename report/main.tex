\documentclass[11pt,
a4paper,
oneside,
abstract=true,
bibliography=notoc,
numbers=noenddot,
toc=nolistof
]{scrreprt}
\usepackage[utf8]{inputenc}
\usepackage[OT1]{fontenc}
\usepackage[default,light,bold]{sourceserifpro}
\usepackage[light,bold]{sourcesanspro}
\usepackage[regular,bold,scale=0.9]{sourcecodepro}
\usepackage[american]{babel}
\usepackage[babel=true]{microtype}
\usepackage{amsmath,amssymb,amsfonts,amsthm,mathrsfs}
\usepackage{graphicx}
\usepackage[nospace]{varioref}
\usepackage{caption}
\usepackage{array}
\usepackage{booktabs}
\usepackage{enumitem}
\usepackage[autostyle=true]{csquotes}
\usepackage{tikz}
\usepackage{minted}
%\usepackage[backend=biber,bibstyle=ieee,citestyle=numeric-comp]{biblatex}
\setminted{fontsize=\small}
\setminted{autogobble}
\setminted{linenos}

%% Paragraph control
\clubpenalty=10000% excludes orphans
\widowpenalty=10000% excludes orphans
\displaywidowpenalty=10000% excludes widows
\setlength{\parindent}{0em} % indent on first line of a paragraph
\setlength{\parskip}{1em} % vertical distance to preceding paragraph
\linespread{1.0} % note the strange values: single spacing: 1.0, 1.5 spacing: 1.3, double spacing: 1.6

\usepackage{fancyvrb}

% Make document internal hyperlinks wherever possible. (TOC, references)
% We load this package last just to be safe. Otherwise it becomes a diva.
\usepackage[breaklinks,linkcolor=black,colorlinks=true,citecolor=black,filecolor=black]{hyperref}
\expandafter\def\expandafter\UrlBreaks\expandafter{\UrlBreaks%  save the current one
   \do\a\do\b\do\c\do\d\do\e\do\f\do\g\do\h\do\i\do\j%
   \do\k\do\l\do\m\do\n\do\o\do\p\do\q\do\r\do\s\do\t%
   \do\u\do\v\do\w\do\x\do\y\do\z\do\A\do\B\do\C\do\D%
   \do\E\do\F\do\G\do\H\do\I\do\J\do\K\do\L\do\M\do\N%
   \do\O\do\P\do\Q\do\R\do\S\do\T\do\U\do\V\do\W\do\X%
   \do\Y\do\Z\do\0\do\1\do\2\do\3\do\4\do\5\do\6\do\7\do\8\do\9}
   
% Syntax containing underlines and other symbols that LaTeX thinks to be math symbols need to be in a verbatim environment
\newcommand{\sytx}[1]{\mintinline{text}{#1}} 

\title{Flake OS\\\hspace{1em} \large AOS Final Report Group 7}
\author{Altin Alickaj \and Thierry Backes \and Florian Bütler \and Manuel Hässig}

\begin{document}

\maketitle

\tableofcontents

\chapter{Introduction}
We present our project report for the operating system we have been building for the last couple of months. 
In the final submission, we present a system describing our approaches to the milestones M0-M6 as well as the 
following individual milestones: A file system, networking, a name server and a shell.

Together, these milestones build the operating system we call Flake OS. Why is it called that?
Well --- sometimes it is a little bit flaky. While the name currently can be seen as a wordplay rather than a serious name, given more time, the system could become more elegant and sophisticated - like a snowflake, perhaps?

\chapter{Simple Shell and Power Management}
For the first milestone, we want to briefly outline the problems we faced,
the solutions we found, and the approached we followed. 
While there is not much to tell, we still want to provide a short account of this introduction into the project.

Of course, note first that we use QEMU in this milestone and it's provided UART interface.
In the specification online, we found everything necessary for a successful communication with the interface.

\section{Output}
When wanting to print a character to the screen, we wait until the graphical device is
ready and then we send it to make it appear on the screen.

\section{Input}
Input works in a similar fashion. We wait until the device says there is data to
be read and then we read it into local memory.  On the first try, we made a mistake by
confusing \mintinline{c}{0x4} with \mintinline{c}{1 << 4} and hence the third bit was checked instead of the
fourth. This let the program make progress even if there was no character,
resulting in bad performance.

\section{Simple Shell}
Implementing a simple shell requires to store every character that is entered
until the return key is pressed. This is done in a char array. The char array
allows a maximal input length of 255 characters as a first choice.  Once the return key is
entered, the string (stored in the char array) is terminated and evaluated. If
it matches an implemented command, it executes it.  As \texttt{printf} uses the function
we implemented for output, we were able to use it quite simply to print whole strings to the screen. 
We provide a \texttt{hello} and \texttt{help} command that just print some text back to the user,
otherwise the input is ignored. We also implemented the functionality of \texttt{Ctrl+C}
to abort an input. Moreover, there is the functionality of a backspace implemented
as well. This is done by moving the cursor one character back, printing a space, 
i.e. an empty character and then moving the cursor back again. To make
things a bit more aesthetic, a \texttt{\$} is printed to start an input and a newline is
started when an input is evaluated.

\section{Shutdown/Reboot}
Shutdown and reboot is all about reading the specification. We decided to use the
SMC32 calling convention and a fast call. First, we had trouble understanding what
the value of the service call range was. But it turned out it is the "Owning
entity number", i.e. 4. We also had a bug that was confusing for a while, but it
turned out, we were just so focused on getting the value right that we forgot to
call the \mintinline{asm}{smc} assembly instruction, therefore the value in the register was never
evaluated. The difference of shutdown and reboot is just an offset of 1 of the
value stored in the register. The most challenging part of this task was to
understand the specification correctly and getting the assembly instructions
right.

\chapter{Memory Allocator}

\section{Physical Frame Allocator}

\subsection{Algorithm}

I first wanted to implement a buddy allocator and implemented a reference
implementation for testing purposes outside of barrelfish and then imported that
into barrelfish as my memory allocator. Unfortunately, I struggled understanding
the concept of how capabilities are split i.e. I got that I get a new capref but
it was not clear for me what happens to the rest. Do I get another new capref
with that? After some time I figured out that it the original capability is
still untouched from the point of view of the memory allocator and the only
difference is that when I try to split the same part, I get am error from the
underlying capability management system(correct name?).

Also havent I decided how to handle the noncontinous memory regions that are
handed to the memory allocator nor how to handle memory regions that are not a
power of two in size.

All this then lead me to implement a simple first fit allocator with a doubly
linked list to get things started. Later I improved the first fit allocator to a
next fit allocator with a circular buffer.

There were more obstacles to overcome: I was puzzled first how to allocate an
object on the heap .i.e as you do with malloc in normal C programs. I then
figured out that this is the purpose of the slab allocator. There I wasnt sure
what block size to choose and how much memory the slab allocator needs. So I
went over all object I wanted to allocate and choose the biggest object in terms
of bytes as the block size and just gave the slab allocator a bunch of bytes.

As I needed a list implementation for my first attempt of implementing the
memory allocator, I added my own to the source tree. Little did I know that
there is already an implementation under collections/. However, this one anyways
uses malloc so I could use it at this stage, but still useful to know for later.
I later changed my memory allocator to natively be implemented with links so I
didn't need my list implementation anymore.

In the following I am going to discuss my implementation of a next fit memory
allocator.

(maybe explain the next fit allocator?)

\subsection{Datastructure}

As mentioned above, the datastructure for my next fit allocator is a circular
buffer.  The circular buffer is implemented as a doubly linked link list and the
smallest unit is a mmnode (memory manager node).  Each mmnode represents
continous region of memory and is linked to the next node that is represent the
next closest region of memory. The memory nodes that represent the start (lowest
memory address) and the end (highest memory address) of the memory are linked
with each other.  A node can be split into two nodes to accomodate its size to
the size of the requested memory or be merged with one of its neighbours, if
they are free.  A mmnode contains its type, wheter it is free or allocated,
pointers to its adjacent neighbours, its memory base address, its size and a
capinfo. The capinfo contains the capref and its original memory base address
and size:

\begin{lstlisting}
enum nodetype { NodeType_Free, NodeType_Allocated };

struct capinfo {
    struct capref cap;
    genpaddr_t base;
    size_t size;
};

typedef struct mmnode_t {
    enum nodetype type;
    struct capinfo capinfo;
    struct mmnode_t *prev;
    struct mmnode_t *next;
    genpaddr_t base;
    gensize_t size;
} mmnode_t;
\end{lstlisting}

The datastructure is concluded by storing a pointer to the head of the linked
list. As it is a circular buffer, it does not really have a natural head, but
this is the pointer to the current position in the circular buffer. The memory
allocator has more in it and they will be discussed at a later point.

\begin{lstlisting}
struct mm {
    struct slab_allocator slab_allocator;
    slot_alloc_t slot_alloc;
    slot_refill_t slot_refill;
    void *slot_allocator;
    enum objtype objtype;

    mmnode_t *head;
};
\end{lstlisting}

\subsection{Add memory}

Initially, the memory allocators head is NULL as there is no memory at the
moment available. The memory allocator gets hold of new memory from the init
process that reads that information from the bootinfo. With each new memory
region added we add another node representing this region. The memory region
comes in the form of a capref to a RAM capability. By inspecting it, we can read
its memory base address and size and create a node for it. As its our first node
its next and prev pointer point to it self and the head points to it.  As more
memory regions are added, more nodes are created and inserted before our current
head. As it may be clear we mark all these nodes as free memory.

\subsection{Alloc memory}

If any other component in barrelfish wants a ram capability, it needs to ask us.
After some initial sanity checks of the request, we now need to find a memory
region that fits the request. There are multiple parameters such as size of the
memory and its alignment i.e. it has to start on a address that is a multiple of
the alignment.  To fullfill these requirements we traverse the circular buffer
until we find a node that is marked as free and its size is large enough. If the
memory that is represented by the node is not aligned as requested, we check if
the aligned version still fits in this node and continue our traversal
otherwise.  If the node is not aligned but is still big enough, we split the
node at an aligned position into two nodes with the right one being alinged.
Splitting just inserts the new node into the circular buffer by pointing the
previous node to itself and itself to the next node. Of course we need to update
the base addresses and sizes of the nodes.  As we now have a node that fulfills
all requirements we retype the capref to the correct size for both nodes and
storing it as the result.  This is also the node we are going to start our
search from for future memory allocation request (i.e. next fit).

\subsection{Free memory}

Freeing memory reverses an allocation. We identify the correct node by
inspecting the provided capref, that gives us the base address and size.  If we
did not find the node we did not hand out this memory region and ignore the free
request.  If we found the node, we destroy the capref and marke the node as
free. Further, we check both neighbours for a possible node merge. In a node
merge we check whether they represent also physical adjacent memory regions to
our memory region. This is crucial to not try and merge the "first" and "last"
node as our data structure is a ciruclar buffer but the memory is still linear.
Also the neighbours have to be marked free as well. If that all is the case we
merge them by simple pointing the previous node to the next node.

This concludes our memory allocator.

\subsection{Partially free memory}

I have also implemented to free partial memory. We know check the incoming free
request, whether its address is represented by a node and if yes and the size is
not the whole node, we split the node according to the partial free request.
Here we have 3 cases: left aligned, middle aligned and right aligned partial
free. For left and right aligned we need to split once and for the middle
aligned we need to split the node twice. The rest works as usual afterwards as
the normal free, the only difference is that we work on a new node.

\subsection{Slab memory allocator}

Our mmnodes need some memory as well store its data. But where to take it from,
when the nodes have the be created first before memory is handed out by the
memory allocator?  This is the purpose of the slab allocator. It can hand out
memory but only of a certain block size. For the slab allocator instance used in
the memory allocator its block size is the size of a mmnode. We initially give
the slab allocator some memory from the init process and if that runs out we
have to get it from the actual memory allocator that should now be bootstrapped.
The slab allocator is used in a node split to create a new node and also when we
add some memory to the memory allocator. We return a slab on a node merge. We
also need to periodically check, whether we need to refill the slab allocator
with new memory. We do this after after each memory allocation in our memory
allocator.  Why do we even need to do that?  Well, once the slab allocator has
run out of memory and only then want to allocate more memory to it we need to
ask the memory allocator for memory but this one needs a slab to allocate
memory. So you see the problem here?

\subsection{Slot memory allocator}

To store the capref we hand out for every memory allocations we need some memory
as well. Initially it also gets some memory from the init process.  We allocate
a slot for each capref we return for an allocation request and free the slot on
a free request (at the moment, however, the slot allocator does no bookkeeping
for allocated slots).  This one also needs to be refilled and we check that
before each slot allocation. This one is even more tricky, if we run out of
slots. A slot refill needs to allocate memory, that needs a slab for its
bookkeeping that may trigger a slab refill and that requires a slot too.  For
this purpose, I also added a function that reports the number of free slots.

\section{Frame mapping}

\subsection{Page table}

Next, we need to be able to map a virtual address to a frame capability that was
derived from a ram capability that represents a pyhsical memory region.  For
this prupose we need to create a page table that is able to do this mapping.  We
have a 4 level deep page table. The first level (L0) is at a well known location
and exists already. So we only need to create the other levels on the fly as we
need them i.e. need to store a value in one of its entries.  Once we reached the
last level, we store the frame capref and are done.  The datastructure here is a
tree. Each page table stores pointers to their children.  The memory to store
the datastructure is once again done by a slab allocator (not the same one) with
some memory provided by the init process.

\subsection{Slab allocator refills}

Now, we are able to actually refill the slab allocator.  Here we first allocate
a frame capref (derived from a ram capref) and map the frame into the page table
for the given virtual address. We can then tell the slab allocator that there is
some more memory at the virtual address for its use.

\section{Tests}

To test all the above discussed functionalities I wrote some tests. They include
alternating memory allocations and frees of one base page i.e. 4KB (the
iterations may vary between 8 and 512 and alignment of 1 or 4KB), consecutive
memory allocation and then frees (again different parameters), provoking many
node merges by first allocating many regions and then free every second and
finally the rest. There are also test for frame mappings, slot and slab
allocator refills and a test where the requested size is exponentially increased
until there is no memory left to fulfill the request and then freeing all
memory.  During this, there are page tables created that are at the moment not
freed.

\section{Obstacels}

\begin{itemize}
\item Capref retyping
\item pointer arithmetic
\item checking if a page table entry is free before inserting a new mapping
\item not corrupting the memory manager by doing a slab allocator refill at the wrong time
\item avoiding resursive slab allocator refills
\item the huge complexity of the slimmed down version of barrelfish
\end{itemize}
\chapter{Processes}

\section{Paging}

For each page table we store the information in the following struct:
\begin{minted}{c}
struct page_table {
    struct capref cap;  ///< cap that represent the memory where this page table is stored
    struct page_table *entries[PTABLE_ENTRIES];  ///< the entries of the page table
    struct capref mappings[PTABLE_ENTRIES];      ///< the mapping of the page table
    genpaddr_t paddrs[PTABLE_ENTRIES];
    uint16_t filled_slots;  ///< nr of filled slots in this table
};
\end{minted}

Since a higher-level entry might point to multiple lower-level ones, each page table entry has a list of possible children \verb|struct page_table *entries[]| and keeps track of the number of filled slots.

This state is stored in the \verb|struct paging_state{}|.

\subsection{Allocate free regions of virtual address space}
Tracking virtual memory is identical to tracking physical memory. Therefore, we reused the \verb|mm_tracker| from the physical memory management. 

A call to \verb|paging_map_frame_attr_region()| first asks the memory manager for a free region of virtual memory, then maps this region into the page table. To map a region, we walk down the page table and create entries on each level if they don't exist.

\subsection{Map large frames}
Mapping large frames is very similar to mapping single frames, since they can span multiple L3 entries but only a single L2 or L1 entry. Therefore, we iterate over the virtual address in increments of \verb|BASE_PAGE_SIZE| to create a new L3 entry for each one and append them to the L2 array.

\subsection{Morecore}
For allocation in morecore, we simply reserve the specified size using our memory reservation datastructure for the heap.
One problem we've experienced is that our datastructure can only reserve multiples of page sizes.
As a result, we've allocated a whole page when even small requests of a few bytes are done. This lead to extremely many
page faults, slowing down our system noticeably. As a solution, we always keep the last allocated page cached in morecore,
and if subsequent requests fit into the remaining cached memory region, we return that. If that's not the case, we use
the reservation datastructure again to reserve new heap space. With this fix, our page fault counts reduced drastically,
giving our system a more natural speed.

\subsection{Unmapping}
Unmapping a region consists of two parts: The entries in the actual page table need to be removed, and the state of the virtual memory tracker needs to reflect this change. The first step is to acquire the node in our memory tracker responsible for this memory region as it stores the capabilities of the page table.

A region might span multiple L3 entries, but only a single L2 and L1 entry. Therefore we can traverse the region in increments of \verb|BASE_PAGE_SIZE| and remove the corresponding L3 entry on each iteration, and the L2 and L1 entries during the last. After destroying all the capabilities, the region is marked as free in the memory tracker and can be reused.
However, if a L1 or L2 table doesn't even exist, we don't bother looking any further for L3 tables that need to be mapped and can proceed to the next one.
Thus, we pay a heavy price only if we land at a sparse L3 table.

Subsequently, we started working on a better version, which included storing mapped regions in a linked list using our memory region tracker we've
used throughout the previous milestones. That way, we could directly move to the regions of interest. However, due to time constraints, we didn't get
to use it in code. The implementation of this data structure for tracking mapped region better has been left in the codebase for review if interested.


\section{Process Creation}

\subsection{Load from multiboot image}
 We did not implement the ELF loading from the filesystem due to time constraints. Thus, we're only able to load multiboot binary. So we need to find our binary by looking for it by its filepath in the boot image first.  Once we found it we start setting up our bookkeeping for a spawned process. Here we set the name of the process to the binary name and store the module location that holds the binary.  We also need to load the binaries arguments from the boot image as they are also hardcoded in there. We get them as a raw string and hence they need to be parsed. The chosen argument separator here is a whitespace. If there are multiple whitespaces between the arguments, they are stripped.  As we later want to load binaries not only from the boot image but also from the filesystem, we moved the following part into another API endpoint so the common functionality to actually spawn a process can be shared.

\subsection{Find ELF image}
We will now refer to the process that spawns a process as \emph{parent}, and the spawned process as \emph{child}.  To work with the binary we need to map it into the parents virtual memory space. To do this, we map the address of the module into the vspace with the help of our paging infrastructure. To check whether we mapped the module correctly we can try to access the first four bytes of the mapped address in the parents virtual memory spaces. As we mapped an ELF binary the first four bytes should correspond to  the ELF magic bytes \verb|0x7f|, \verb|E|, \verb|L| and \verb|F|.

\subsection{Create intial CSpace} 

Next we need to setup the capability space of
the child. It has the well known layout of one root L1 Cnode and then multiple
L2 CNodes in some predefined slot of the L1 CNode.  So first we create the L1
root cnode. From this we can create the L2 CNodes: task cnode, three slot
cnodes, the base page cnode and the page cnode.  The task cnode holds multiple
capabilities such as the dispacher, dispacher frame, argument page, the endpoint
to itself (created from the dispatcher capability) and the root cnode from
above.  At this point we also copy the parents endpoint to a well known location
in the child cspace such that the child can use it later to setup a channel to
the parent for inter process communication.  The three slot cnodes are empty and
contain space for the child's initial slot allocator and more if needed.  Each
slot of the base page cnode hold a ram capability of the base page size such
that the child has some initial memory to work with.  Finally the page cnode has a capability to the top level page table in the first slot. The other slots can
be used to store other page tables.

\subsection{Create initial VSpace} 

In the child's virtual memory space we need to create a L0 page table and store it in the page cnode. We also need to copy the L0 page table to the parent virtual memory space so we can invoke it. This is needed to setup the paging infrastructure for the child. If we would not map it into the parents vspace we wouldn't have the right to write to it.  As already mentioned above, we need to store empty ram caps in the base page cnode, therefore we need to allocate them and store them in the corresponding slots.

\subsection{Load the ELF Image} 
Now that we have a working virtual memory space
in the child, we can parse the ELF binary and load the segments in the childs
virtual memory space. This work by defining a callback function that is called
for each segment that is encountered in the ELF binary.  In the callback
function we allocate first a frame in the parents vspace and map the segment
frame, for which the callback function was called, into the frame. We also need to
translate the access rights from the ELF binary segments to the virtual memory
space as they are be different. We first map it into the parents vspace again to
be allowed to perform operations on it. After the parent is finished, the final
frame is also mapped into the child's vspace.  Finally, the binary is parsed for
the global offset table header such that we can initialize the child properly.

\subsection{Adding a dispatcher} 

To setup a dispatcher we first need to allocate
a dispatcher frame. This frame is used by the CPU driver to store information
about the process. Here, we again map the frame into the parents vspace and the
childs vspace.  To finish things up, we setup the dispatcher fields and put
initial information in the dispatcher frame such aus core id, virtual address of
the dispatcher frame in the child's vspace, the process name and the program
counter and tell the child to start in the disabled mode.  Finally, we
initialize the offset registers and disable the error handling frames.

\subsection{Set up arguments} 

The child process also needs to know with what
arguments it has been invoked with. For this, another frame is allocated and
mapped into both vspaces.  The frame is expected to have a specific layout:
First, a struct with some meta information of the frame such as at which
address each argument is located.  This struct is followed by the actual
arguments until it is finished by a null pointer.  The child process expected
its first argument in the enabled save area to contain a pointer to the above
mentioned struct, so we set that at the end.

\subsection{Start the process} 

Invoking the dispatcher is the easy part, namely
calling a sys call with the correct arguments we just set up.

\section{Process Management}

\subsection{Datastructure} 

We chose a simple linked list for storing the
currently running processes. The head is well known with the init process being
the first process in the system.  All other later spawned processes are appended
to the linked list.  Each process gets a PID assigned, after the process was,
invoked by checking whether the current PID counter is already used and incrementing it until a free one is found.

\subsection{Kill a process} 
Killing a process identified by its PID now boils down to traversing the linked list until we find the matching process, stopping its dispatcher and removing from the linked list.

When a process terminates the function \verb|void libc_exit(int status)| is called. Before terminating the thread, we are calling the \verb|aos_rpc_kill_process| function to stop the dispatcher and remove the process from our linked list.


\section{Pitfalls} 

We had some serious trouble with refills during a page
mapping. Namely, when we ran out of slots/slabs and attempted a refill, it may have 
happened that the refill actually used the page we wanted to map, resulting in an
error. We solved that by ensuring that no refills happened during an ongoing
mapping.

Another interesting bug occurred during tests of spawning many processes. The first four processes spawned just fine but when trying to spawn the fifth process, the whole system collapsed without an obvious error. After some debugging, it turned out we stored the metadata of a spawning process in the test on the stack and with 4 processes the stack was full and the init process terminated. We solved that by storing the metadata on the heap instead.

% other pitfalls:
% virtual address starts at 0 instead of VADDR_OFFSET
% overwrite string termination of args
% reuse same datastrucuture for process creation

\chapter{Local Remote Procedure Call}
\section{Introduction}
% give a short introduction into the problem, the goal, the approach,
For intra-core communication, our OS builds on top of the already provided Lightweight Message Passing (LMP).
Our communication is done via a framework we have built on top of bare LMP channels.
It is part of a wider abstraction of our general remote message passing (RPC) framework,
which works by combining LMP and inter-core communication, which is achieved using user message passing (UMP).

The resulting RPC framework abstracts those two frameworks under a single message passing interface
for any process on any core. In this chapter, we only talk about LMP. Any mention of RPC
in this chapter concerns only the abstraction of LMP in our codebase.

\section{Where is it used}
We use LMP channels for every newly spawned child process. Every process creates 
establishes one LMP channel to the memory server, one for the init-server (which handles e.g. spawn requests),
and one to the serial driver. Later one, we will see that our nameserver, which allows the registration of 
services also uses LMP for communication to clients.

\section{Designing The Message Passing Infrastructure}
% Use cases: memory, init-comm, serial driver, between processes
% Either async vs sync -> easily usable for a server as well as for a client
% Choices: All using same channel. Locking -> where is concurrency possible
%          Separate channels for separate uses; also: how is state passed in messages
Lightweight Message Passing (LMP), which is used to communicate between processes
on the same core is merely a means of communication. On top of such a functionality,
various designs can be implemented for allowing a safe, reliable and usable system
for communication. We will see in this chapter how we've built a system which allows
an easy integration for use in a Memory Server (to distribute RAM capabilities 
to processes), a Serial Server (for input/output to a screen) as well as an Init Server 
(for communication to the \texttt{init} process).

Of interest are the different choices one could use LMP to communicate asynchronously and/or 
synchronously. Also, it's of big interest to utilize the asynchronous nature of LMP to
achieve high concurrency during operation. Lastly, while a functioning protocol for the basic
functionalities required (communication with init, memory and serial servers) 
should surely be logically separated, there are many options how one might separate them physically.
These can be isolated components over different channels, or handled by a single channel all-together.

\subsection{Blocking vs Non-Blocking}
To embrace the non-blocking nature of LMP achieved through upcalls, we provide an 
easy interface for a process to register for the reception of such messages. This interface is 
access through the function 
\begin{minted}{c}
errval_t aos_lmp_register_recv(struct aos_lmp *lmp, process_msg_func_t process_msg_func);
\end{minted}
which is defined in \texttt{aos/aos\_lmp.c} for registering a server.

\subsubsection{Non-Blocking}
For non-blocking communication, we focused mostly on the use-case of arbitrary servers, 
which offers services to be received and processed in an endless-loop.

Since any server (e.g. the memory server) can register such a handler, it becomes trivial to 
implement servers. If registered through the interface, the framework automatically takes care of 
re-registering the process function after the reception of any request. 
The only responsibility we give to the server is then to actively wait for 
messages to come in, which can be done in the all-too-familiar manner:

\begin{minted}{c}
    while (true) {
        err = event_dispatch(default_ws);
        if (err_is_fail(err)) {
            // handle error
        }
    }
\end{minted}

\subsubsection{Blocking}
When talking about blocking communication, we have mostly client-applications in mind. 
While a server usually only responds, a client requests.

In our cases throughout the course, these requests of clients made sense to be blocking, 
so that user-code only continues to run after a response has been received.

Our approach to implementing blocking LMP has been one of the greatest downfall in
our operating system and has been proven to be very much against the idea of the strengths of LMP.

Blocking communication is achieved by polling on the channel until a message has been received.
As stated, having send-then-receive patterns in mind, we provide a call for doing 
both at once:

\begin{minted}{c}
errval_t aos_lmp_call(struct aos_lmp *lmp, struct aos_lmp_msg *msg);
\end{minted}

Using this interface, a client can send a request inside
\begin{minted}{c}
    struct aos_lmp_msg *msg
\end{minted}
and then await the response, which is stored in a field of the \texttt{lmp} struct above.

The question that the reader familiar with the LMP infrastructure of Barrelfish might ask is
why we decided to poll on messages for receiving responses on client requests.
In an older design, we implemented blocking communication through receiving the message asynchronously 
by registering a receive handler and just waiting for a response using \texttt{event\_dispatch()} again.
However, due to race conditions between different client calls, we abandoned this idea for the time being. 
While the polling worked as a temporary solution, we never got to the point where a rewriting of the client receiving
logic was possible, especially since the integration of our individual projects led to many critical bugs of higher priority.

Further, we only fully understood stack-ripping and methods to call manually stack-managed code
into automatically stack managed code one week before the deadline. It was only at that time that
we also learned about polled waitsets.

We're still convinced that this improvement would lead to a performance increase in 
our implementation, especially under heavy load.

\section{Binding}
Before using a channel, it needs to be created. 
In our case, we needed to create channels between the \texttt{init} process and a newly-spawned
child process. After the channel has been established between the two parties, 
it can be used for communication. During the implementation, we were very confused by the notion of
endpoints, waitsets and what a channel actually embodies. The book alone did not help
us understand the structure and process well enough. Through a lot of trial and error,
we later successfully managed a first request.
In the following, we describe how channels are established.


Let's have a look at the init server.
When a new child process is spawned, the \texttt{init} process hands it an endpoint-capability
to itself. That way, the child already knows which channel it needs to bind to. 
Given that capability, the child has enough information to contact \texttt{init}, and 
\texttt{init} has enough information to receive messages. To also allow \texttt{init}
to send to the child, the child process will create a new endpoint and send it to \texttt{init}
in a binding handshake. Using this endpoint, both parties can then send and receive.
At this point, the channel has been fully established.


\section{RPC State}
During operation, we support sending messages of dynamic size.
Supporting the option to send messages without causing page faults was crucial 
to us. Besides increased performance, it is also necessary for being able to use our memory server.
During page fault, we might need to receive additional memory. However, we are not allowed to cause another
nested page fault in that. For that reason, we added a static buffer to each RPC channel of 4096 bytes, 
which is used to send statically sized RPC messages that fit into it.
That way, functionality like sending RAM over RPC can use an already-mapped memory region for 
receiving messages, thus avoiding page faults. In general, any established 
RPC channel consists of the following components:

\begin{itemize}
    \item A buffer allocated at spawn time for receiving messages smaller than a page, which guarantees
    communication without causing page faults
    \item Knowledge for each message a channel is receiving indicating whether we may allocate a large
    buffer for receiving messages larger than one page, i.e. we may incur page faults using this channel.
\end{itemize}

We now describe the state which is kept by our RPC communication protocol.

\subsection{Message Sizes}
We support sending messages of dynamic sizes. To achieve this, we considered 
either shared memory or breaking messages into smaller packets.
While shared pages will decrease the overall LMP-traffic, we would also 
require additional logic for sharing the pages necessary as well as 
copying a potentially large source buffer to this large memory region.

Many questions arise:
\begin{itemize}
    \item Who has read/write access to the shared page? 
    \item When is this shared page unmapped?
    \item At which size of payload will shared memory make sense?
\end{itemize}
On the receiving side, the data would need to be copied out of the shared region 
into newly allocated memory. The copying is necessary since the sender 
might want to unmap the shared memory region at some point and it shouldn't be 
the case that both processes modify the same shared memory region.

Splitting messages into multiple parts has a very straight-forward structure 
without having the additional bookkeeping complexity shared pages have.


In the end, we decided to use the message-splitting approach.
More precisely, any message packet we send over LMP is sized at most 32 bytes
large. This block size was chosen arbitrary and it would have been interesting to measure
to compare performance with larger block sizes.

\subsection{Message Structure}
Below, we describe the structure of a whole message of dynamic size.
\begin{minted}{c}
struct aos_rpc_msg {
    aos_rpc_msg_type_t type;
    char *payload;
    size_t bytes;
    struct capref cap;
};
\end{minted}
Every message sent over an LMP channel contains a message type determining 
the type of request, the actual payload, payload size and optionally, 
also a capability to be transferred through the channel. 
Additionally, to the beginning of a message to be sent, we prepend its size 
to the payload. That way, the receiving size can determine the size of an 
incoming message and knows how many more are to be expected.

\subsection{Invariants}
To achieve thread-safety, our LMP communication protocol 
follows the following invariants:
\begin{itemize}
    \item Client send-receive patterns are atomic with respect to the channel
    \item Server receive-send patterns are atomic with respect to the channel
    \item No process uses a channel as a server and client at the same time
\end{itemize}
These invariants allow us to avoid race conditions and the guarantee that 
when a client receives a response, it will be the response to the client's
request and not the response to any other request from another thread.
To achieve that, it's also necessary that no process uses a both server and client at the same time.
With that at hand, we have a total order of RPC requests of a client and no 
response-mixups, where a client receives a response which is meant for another thread using the same channel.

Note at last that these invariants are only concerning a single channel.
Different channels follow no requirements with respect to other channels 
and can be treated separately.

\section{Implementing Functionality on Top of LMP}
Now that the structure of our use of LMP-channels have been explained, 
lets have a look at how they are used to build actual services and how 
they interleave at the example of our memory, terminal and init servers.

\subsection{Physical Separation of Services}
By our invariants, send-receive and receive-send  patterns must be atomic. Thus, we can't allow an RPC call to 
start recursive RPC calls (which would be breaking atomicity). Since dynamically-sized messages 
need to allocate dynamic memory when receiving an arbitrary-sized response, 
it could otherwise happen that memory is requested during an RPC call.
This could (and did) lead to recursive page faults and other unexpected behaviour.
Further, a channel that is registered a waitset cannot at the same time be
used for blocking calls as the received message will also result in an upcall.

Therefore, each process gets two different channels to the core-local init process:
one to register in the waitset to receive requests from init (the server channel) and
one to send (blocking) requests to init (the client channel). Further, we decided
to establish a separate memory channel with flags such that it will never
enter a code path incurring a page fault.

Of all these channels only the server channel is registered with the default
waitset as the other channels are only used in a blocking manner.

\subsection{Memory Server}
The memory server runs on the \texttt{init} process and receives RAM cap requests.
Since these requests and responses are all statically sized, no recursive page faults 
happen. Since our interface allows sending/receiving capabilities, this 
is already enough to implement the memory server.
Similar to the \texttt{init} server, it performs a handshake with the child process
when it is spawned. 
\subsection{Terminal Server}
The terminal server runs on its own channel and communicates with the UART driver, which will be explained in the Shell chapter.
\subsection{Init Server}
% TODO
The init server registers the following handle-function 
\begin{minted}{c}
    errval_t init_process_msg(struct aos_lmp *lmp)
\end{minted}
It is responsible for handling spawn-requests, sent strings/numbers, PID lookups to receive 
the name of a process given its PID, and clients can also request to receive a list of all PIDs.

All requests to this service are stateless - all information for a request must be 
contained in the request itself.


\section{Encountered Problems}
We had a few problems with our initial RPC implementation, 
which after some revisions led to the invariants stated above. Some of these problems were:
\begin{enumerate}
    \item Mixing server and client functionality led to race conditions.

    \item When implementing support for page faults, we realized that it must be possible 
to guarantee that a RPC call will cause no recursive page fault.

    \item When mixing LRPC with URPC, it has become crucial to abstract LMP and UMP as much 
as possible, to avoid a confusing interface for a programmer in user-space.
\end{enumerate}

\section{Transfer Speed}
We plotted the transfer speed of our LMP channel for different string sizes by using \texttt{aos\_rpc\_get\_string}
and measuring the total time.
\begin{figure}
    \centering
    \includegraphics[scale=0.7]{lmp_bandwidth.png}
    \caption{LMP Bandwidth}
    \label{fig:my_label}
\end{figure}
As we can see, we reach transfer speeds of up to 1.8GB/s, which is roughly 10 times lower than the usual DDR3. The overhead of the first message which needs to be received can be seen by the lower bandwidth for smaller sizes, however the performance soon is saturated.

\section{Analysis} % Problems; performance; hindsight
Given more time, we would be interested to play with different packet sizes. As stated,
we currently use packets of 32 bytes. Benchmarking performance with other sizes 
would be a good idea. However, we weren't too concerned with this, since most of the RPC
messages we send (and crucially, the ones which are sent most frequently) fit into a single packet.

Another improvement would be to look at rewriting blocking reception of RPC messages,
which currently uses polling to check if an incoming message exists. As said, we 
abandoned the idea of using upcalls for receiving responses to blocking client requests due to race conditions 
which resulted from failure to abide by our stated invariants. Given more time, 
this would be a good performance optimization and a crucial improvement to our system.

Avoiding the idea of message-splitting altogether, 
using the shared-memory approach for dynamically sized messages could be 
analyzed further and performance compared. We suspect that starting from some size, 
sharing memory will outperform message-splitting substantially.


% extra:
% big messages
\chapter{Virtual Memory}

What should a virtual memory system provide? memory, safety, flexibility, efficiency
The next sections will describe how we achieved these goals.

\section{Memory Layout}
Central to this milestone was coming up with a useful layout for the virtual memory. To this end
we divided the virtual memory space available to user space (the lowest $2^{48}$ addresses on ARMv8)
into an unusable, a read-only, a heap, and
a stack regions. The unusable region is the page with the lowest addresses -- basically everything
that results in a segfault. Above that we have the read-only section where the code and all the static stuff %TODO: specify static stuff
is mapped.

The stack region grows down from the highest user space address. We made the design
choice to reserve a fixed amount of memory for each thread. This allows us not to worry
about stacks growing into each other and instead having a predictable stack overflow.
Currently, this value is set at 1 GiB per stack (i.e. thread), which is sufficient for our devboard.
However, for other platforms it can easily be increased as there is plenty of virtual memory.
Since the number of threads in Barrelfish is limited (see threads.h, currently 256), we
reserve the top 256 GiB for the stack region.

The heap region starts after the read-only section, which ends after 512 GiB and stretches all the way
up to the start (or rather end) of the stack region.

\section{Allocating and Freeing Memory}
In order to keep track of the address space, we reused our datastructure that we used to track
RAM after some refactoring to have a common memory tracker interface. For each of the three usable
regions described above we have a separate memory tracker to track which parts of the address space
have already been reserved. This allows us to reuse all the implementations for allocating, splitting,
and freeing of memory regions that we already had implemented for RAM.

By separating the management of virtual addresses for these regions into separate memory trackers,
we did not have to worry about the heap growing into the stack and vice-versa as the respective 
memory tracker cannot hand out addresses for other regions than their own (given proper initialization).

It is important to note that requesting memory using \mintinline{c}{paging_alloc} would only reserve a free region
of virtual memory in the appropriate memory trackers. The behavior we settled on was that by default
\mintinline{c}{paging_alloc} would find free virtual addresses in the heap region as a wrapper around
\mintinline{c}{paging_alloc_region} where the caller can select from which memory region the free addresses should come.

These reservations on free virtual addresses are by themselves not backed up by memory. This is only
done once the memory has been accessed and a compulsory page fault has been taken.

\section{Handling Page Faults}
Once our system takes a page fault on a virtual address, the first thing the page fault handler does,
is check which memory region the faulting address belongs to. If it belongs to the unusable region
it goes ahead and throws a segfault.

\section{Dynamic Stack Extension}

\section{Challenges}

\section{Improvements}

\section{Summary}
\chapter{Multi-core Support}

To use another core, we need to boot it, establish a communication channel
between from the old core to the new core and define how memory management is
done across cores.

In our system we make the assumption that core 0 is always present and only
cores 1 to 3 are booted and shut down.

\section{Boot another core}

The main part about booting a core is allocating memory for various
datastructures and passing these memory regions to the new core. Further, we
need to tell the new core, where some fundamental binaries are such that it can
execute them.

\subsection{Create the Kernel Control Block}

We start with the Kernel Control Block that needs to be allocated and retyped.
At this moment it worth to be mentioned, that we always have to pass the
physical memory address as opposed of a virtual one, such that the new core
actualy know where that memory is. 

\subsection{Load the boot driver}

Next, we have to find the boot driver in the multiboot module and map it into
our address space. As the boot driver is nothing else than a ELF binary we need
to find the right entry point, in this case "boot\_entry\_psci". Then we also
need to map the binary into our address space and get its physical address.
Moreover, the bootdriver runs with a one-to-one virtual address to physical
address mapping, so we have to relocate the ELF binary as well.


\subsection{Load the CPU driver}

The same has to be done with the CPU driver, but this time it is a different
binary and with that also a different entry point to be looked for.

\subsection{Allocate the kernel stack}

The kernel needs its own stack, so we allocate 16 countinously aligned base
pages and get their physical address.

\subsection{Loading init}

Also does the new core need to know what initial binary should be executed.
Usually, that would be the "monitor", but as we do not have a "monitor" we use
"init".

\subsection{Allocate kernel memory}

To load the init binary, the CPU driver needs some initial memory, that we have
to allocate for it as well.

\subsection{Initialize the core datastructure}

At this point, we have all the ingredients to boot a new core. We only have to
put all these information together in a compact and well know structure. That is
the purpose of the core datastructure. It contains configuration parameters,
locations and size of various memory regions we just allocated and the core ID.

\subsection{Flusing the cache}

To clean things up, we also need to flush the cache to make sure that all data
we have just written is visible for everyone, including the new core.

\subsection{Spawning the core}

Finally, we can boot a new core by invoking the kernel capability.

\section{Communicate between cores}

The newly booted core needs to fundamental capabilities to work properly. These
include a ram capability that represent a region of physical memory that it can
manage for applications on its core, the bootinfo that contains also the
multiboot module such that the new core can start other processes.  These
capabilities are send over shared memory that is mapped in both cores. We
discuss its communication mechanism further in the next chapter.

\section{Manage memory across cores}

We decided that every core should have its own memory that it can manage. For
this, as we have four cores, a quarter of the memory on core 0 is allocated. The
physical address of the capability representing this memory region is then send
the new core, forged into a new capability and used to initialize the memory
allocator.  Every core has a distinct PID range that it can use for spawning
processes.

\section{Booting all cores and turning cores off and on}

We also implemented both extra challanges. Booting all cores was pretty straight
forward, and we could easily reuse the same logic to boot core 2 and 3.  Turing
a core off and then on again was a bit more involved and neede more digging. But
in the end, we implemented a mechanism that allows core 0 to tell another core
to turn itslef off. Turing a core back on involves the same steps as booting a
core.

% allocate memory
% Create KCB & Coredata datastructures
% load the boot driver and cpu driver
% clean cache
% call spawn

% each core gets 512mb (core 0 initially has all memory and allocs for other cores)
% assumed core 0 is always present (only core 1-3 on/off)
% primary-secondary communication
% pid ranges per core

% extra:
% boot all cores 
% turn core off and on

\chapter{User-level Message Passing}

Communication between cores happens over shared memory. When a core is booted,
the parent core allocates some memory, maps it into its own address space and
provides its physical memory address and size to the child core. The child core
also maps this memory region into its address space, and now both core can
access the same memory region.

With this shared memory, both cores, need to settle on a common communication
protocol such that they understand each other.

\section{Communication Protocol}

Each core can acts as a server or a client. That means it may react on incoming
messages or initiate messages to other cores.  We decided on having a
consumer-producer queue mechanism to exchange messages between two cores, with
the shared memory having the size of two base pages. 

We first implemented the communication protocol with a shared memory size of one
base page, but it turned out that messages, that are sent to a core as a
response to a request and messages that are sent to a core as a request, may
interleave. So we settled to have a shared memory of the size of two base pages
such that one base page can be used for server communication and one for client
communication. 

However, the communication protocol for both server and client is the same. We
will now explain, how two cores can communicate with each other over shared
memory of the size of one base page.

As already mentioned, the communication protocol is based on a producer/consumer
queue implemented with as a circular ring buffer. We need exactly two ring
buffers to communicate, as one ring buffer is for sending a message from on
core, and receiving on the other core. The second ring buffer is the same but
vice-versa, such that both cores can both send and receive messages. 

That means we need to split the base page once again into two equals parts. The
parent core will use the first half of the base page for sending messages and
the second half of the base page for receiving messages. The child core does the
same, but receiving on the first half and sending on the second half. Both cores
keep a pointer to an offset in the ring buffer to know where they should expect
the next message.

\section{Message Format}

Each message in our communication protocol is exactly 64 bytes, i.e. the size of
a cache line. That is important, such that we don't need to flush the message to
the memory, but it is enough to have the message in the cache, for the other
core to see it. This gives us a huge performance gain.

Each message has a header and a payload. The header contains metadata such as
the message type, the message state, payload size and if it is the last messages
in a chain of fragmented messages. The latter is important for sending messages
that are larger than 64 bytes, but more on that later. The header has the size
of exactly 3 bytes, and therefore there is space for 61 bytes of payload for
each message.

The most interesting header parameter is the message state. A message can have
three different states: created, sent or received. When a message is initially
created, it is marked as sent. When a message is sent from the sender's
perspective it is marked as sent, and it is marked as received from the
receiver, when it consumed the whole message.

\section{Sending and Receiving Messages}

To send a message, we need to get our current offset into the ring buffer, i.e.
a particular cache line. First, we check if the previously sent message over
that cache line is marked as sent in its header. If that is not the case, then
we filled the circular ring buffer completely and cannot send another message
until the receiver has consumed some messages. However, if the message is not
marked as sent, but as received, then we can use this cache line to transmit
another message. We copy the message into the cache line, mark it in its header
as sent, and update our offset into the ring buffer to point to the next cache
line. The tricky part here is to deal with ARM's weak memory model. A weak
memory model does not guarantee that stores from one core are observer by
another core in the same order, that they are stored by the former. Luckily,
there are barriers to enforce exactly that.

We need a barrier between checking for the message state and copying the message
in the cache line to ensure that we first check the state being received before
we overwrite possibly not yet consumed messages. And we also need a barrier
after copying the message into the cache line and marking the message as sent,
such that the message is actually written to memory before logically marked as
sent and consumed by the receiver.

To receive a message, it computes the current offset into the ring buffer to
know in which cache line the receiver should expect the next message. It  checks
repeatedly the message stored in this cache line for its message state. Once,
the message state is marked as sent, the receiver got a new message to be
consumed. To consume it, it is copying the content from the cache line, marks
the message as received and updates its offset into the ring buffer.

Again, we need barriers to make the whole receiving process work. Similarly, to
the sender, we need to insert a barrier before consuming a message such, that we
are sure that we first check the message state to be marked as sent before
actually reading its content. Also, we need another barrier between copying the
message content from the cache line and marking the message as received. The
same reasoning as above applies here.

\section{Sending large messages}

At this point, two core are capable of communicating with each other. But only
if the message, they would like to send, fit into a single cache line. We went
the extra mile and implemented the extra challenge fragmentation and reassembly.
For this we need to split the payload, that should be sent, into chunks fitting
into a cache line and reassemble them on the receiver side. Sending chunked
messages is pretty straight forward, but one point to mention here is the case
when we send many chunked messages and completely fill the ring buffer and hence
fail sending the remaining message. In this case, we implemented a linear back
off functionality that retries send the messages up to 32 times, such that the
receiver has time to consume messages and give space for sending the remaining
messages.

We decided to limit the maximal allowed size of messages sent over UMP to two
base page sizes. This way, we can allocate a temporary buffer on the receiver
side, that is guaranteed to contain the whole message after reassembling. The
receiver then simply needs to read all chunked message and reassemble them.

\section{Synchronous communication}

To provide synchronous communication over UMP we also added a function to our
interface that, sends a message and immediately tries to receive one, such that
we could offload this functionality directly into the library.

% maybe talk about server/client and binding

\chapter{RPC}

We now briefly mention how we combined our LMP and UMP frameworks into a generic message passing framework, which
is Remote Procedure Call (RPC).

Essentially, using objects containing a union of an LMP channel and a UMP channel as well as a flag designating what kind of
channel this actually is (UMP or LMP), we were able to wrap around functions of LMP and UMP. Since UMP and LMP operate under 
very similar assumptions and invariants, it was an easy task to implement generic functions which work for both sides.

However, we noticed that sometimes, one just has to resolve to working with the concrete functions (i.e. concrete LMP functions).
This is because handler-functions of LMP and UMP are different. They work and are listened to differently.
We were not able to abstract it away completely, nor did we see a reason to.

This unification required a lot of refactoring, which was a very tedious task. Initially, when implementing a framework
based on LMP, we didn't know that UMP of a later milestone should also be working under the same abstraction,
so our RPC implementation was basically synonymous with our LMP implementation before that.

Please note that we changed the signature of \sytx{aos_rpc_init} due to this unification. Due to the vast differences
in arguments for an LMP init function and a UMP init function we did not find a good way to unify these into one function.
There is a way with a tagged union of argument structs, but we do not consider this to be good design. Since a caller always
knows what kind of channel she is creating, such an abstraction is unnecessary. Thus, we refer in the comments to the
LMP and UMP init functions. In order to implement the grading API we decided for \sytx{aos_rpc_init} to simply wrap
\sytx{aos_lmp_init}.

\section{Shell}

The shell is split into two modules, a serial server and the shell itself.

\subsection{Serial Server}
The serial server can be seen as a rudimentary TTY. It reads and writes through UART and buffers incoming characters.
With only little modification multiple TTYs could be launched, each using a different UART port.


\subsection{Shell}


\chapter{Filesystem}
\section{Introduction}
For the filesystem, we support SD-cards running FAT32. The filesystem we've implemented 
is built as a service, exposing RPC calls for file operations like opening, closing, writing, reading, etc. 
to user processes, which then can be called by the usual libc functions like \texttt{fopen}, \texttt{fclose}, \texttt{fwrite}.

The alternative, which is building the filesystem as a library, has not be chosen because the filesystem as a service in our eyes
gives a better abstraction to the underlying FAT32 operations. Another reason why we chose the filesystem as a service
is because it does decrease RPC communication, leading to overall better performance. Probably most importantly, given processes
access to read/write functionality of the block driver leads to a decentralized filesystem state, where it is not clear how to easily
add thread safety as well as gather information about which file handles are opened where efficiently.

\subsection{Operations}
The operations implemented as a service over RPC calls are:
\begin{itemize}
    \item \texttt{aos\_rpc\_fs\_open}: Open a file with given flags
    \item \texttt{aos\_rpc\_fs\_close}: Close a file
    \item \texttt{aos\_rpc\_fs\_create}: Create a file
    \item \texttt{aos\_rpc\_fs\_read}: Read open file contents into buffer
    \item \texttt{aos\_rpc\_fs\_write}: Write buffer contents to a open file
    \item \texttt{aos\_rpc\_fs\_lseek}: Seeking in a file
    \item \texttt{aos\_rpc\_fs\_dir\_action}: Used to perform the two operations \texttt{mkdir}, \texttt{rmdir}
    \item \texttt{aos\_rpc\_fs\_readdir}: Read directory contents
    \item \texttt{aos\_rpc\_fs\_fstat}: Get file status
\end{itemize}

\section{Block Driver}
After flashing a new filesystem onto the SD-card, we can communicate with it over 
the block driver over the process \texttt{fs}, which has access to the devframe necessary
for using the driver. Since it's the only process with access to the SD-card, we don't have to worry 
about concurrent read/writes in the filesystem.

\subsection{Performance Improvements}
To improve the block driver, we note that we always only read/write blocks of the 
same size with it. Thus, it's enough to only set the block-size once in the beginning,
instead of setting it every time we read/write a block. As we will see during the 
performance analysis in the next section, this will already give a 2x speedup.

There are many other performance improvements which could be made.
Very interestingly to me with respect to FAT32 is that reading/writing whole 
clusters at once could be done much faster. As can be seen in the next section,
sending commands over the block driver alone will take a lot of time, 
so reducing the number of commands sent to the block driver can be of essential 
importance for performance.

It's also possible to read/write specific block-lengths with each write.
However, this change in block-lengths needs to be communicated to the device,
thus another command needs to be send to the SD-reader. We immediately see 
that this constitutes a trade-off between send-commands and DMA-access time.
While always reading large blocks has few send-commands, DMA over large memory 
takes longer than for small blocks. It could have been interesting to also 
check if or at which point it makes sense to work with variable-sized read/write 
commands.

As a last possibility for improving performance, it could have been possible 
to fetch information from the SD-card to obtain more information of frequencies 
which could be used to allow for faster access speed.

\subsection{Performance Analysis}
\begin{figure}[h]
    \centering
    \includegraphics[scale=0.5]{block_driver.png}
    \caption{Read/Write Bandwidth with and without caching}
    \label{fig:block_driver}
\end{figure}

In Figure~\ref{fig:block_driver}, we measured averages of block read and write operations over 50 requests and compared the optimized version with 
the unoptimized one. We can see that the optimization of removing the first command to the SD-card (which sets the block-length)
halves the latency of a read and write request. Furthermore, we see that waiting for the device and the following DMA requests
all take roughly the same amount of time.
In total, the optimized version achieved a latency of 0.239s for 512 bytes, which leads to a bandwidth of 2142B/s. The write bandwidth
takes the same amount of time.
\section{FAT32 Implementation}

\subsection{Assumptions}
We assume we receive a valid FAT32 filesystem. We thus don't do any additional 
checking/fixing/infering of the filesystem present.
We furthermore don't implement thread-safety features into the FAT32 implementation.
Thread-safety will be the task of the filesystem which builds ontop of this.

\subsection{Abstractions}
Wherever possible, we tried to abstract away sectors as much as possible.
Instead, we focused on clusters and indeces into clusters, so that 
we wouldn't have to worry too much about cluster traversals in terms of 
sectors, where in which cluster we are, etc. Seeing FAT32 as a linked-list 
of clusters is much easier than seeing it as a linked-list of sector-arrays.

The sdhc driver uses DMA request, for which we need buffers and their 
physical addresses. We also need to take care that caches are flushed
before reads/writes, to stay consistent.
For the buffers which we use to read/write to, we introduced:

\begin{minted}{c}
struct phys_virt_addr {
    // physical address of memory segment
    lpaddr_t phys;
    // virtual address of memory segment
    void *virt;
    // last sector read/written
    uint32_t last_sector;
    // is the buffer dirty?
    bool dirty;
};
\end{minted}
which we use to keep track of where we read/write to.
The last two members are important for caching, which is another big abstraction we make over the FAT32 implementation
and will be explained later.

\section{Directories}
Directories contain directory entries, holding meta data for it's contained files and directories like names, sizes, data location, etc.
Each directory entry is 32 bytes in size. We didn't implement long file names, so each directory entry has a fixed size.
The FAT32 specification contains many small details which are important. For example, when creating a new directory, all directory entries
of their clusters must be zeroed out. For performance, one could store a flag into the last directory entry, so that none of the remaining
entries in the cluster need to be looked at when looking for particular files. We didn't implement this feature, however concerning the block 
driver speed, this would be a major performance increase for \texttt{readdir}.

Since every new directory contains the two directories "." (dot) and ".." (dotdot), it is trivial to use those for moving back to a parent directory
as well as receive information about the current directory. 

\section{Traversing a Linked List}
Next to reading/writing directory entries, the functionality required to a minimal implementation of FAT32 as per specification is the reading/writing of data clusters containing the data of a file. As known, the data is structured as a linked list of data clusters, where next blocks can be looked up in the 
in the File Allocation Table (FAT). Reading/writing files is then only a matter of traversing linked lists, while continuously reading/writing
the clusters, which represent a single node in the linked list. One has to be careful to assign the EOF flag to FAT entries as well as not overriding
the top bits of a FAT entry, which are reserved. 

Thus, for writing a file, one has to read/write to the directory entry itself (to keep track of the file size), the FAT for reading/writing next-pointers
in the linked list, as well as the data clusters itself. Many sectors are touched during these operations.

\section{Caching}
One file write/read can potentially look at one FAT-sector multiple times,
or could write to the same data sector multiple times. Since every 
read/write from/to a sector on the SD-card is very expensive, a big 
performance increase was the idea of caching reads/writes to the disk, in memory.

In a first attempt, we never performed a read from the SD-card if 
the last read/write to/from the SD-card was to the same sector. This can 
be pretty easily by just remembering the last sector each time.

A next observation was that there is much locality in a directory-entry 
as well as a FAT-sector, since many requests looked at closed-by indeces
of a sector. For that reason, we decided to use a different buffer 
for storing any read/writes to a data-sector and a different one for 
FAT sectors (and directory-entry-sectors). 
A write would still always write, it is write-through.

Using these two different buffer, we benefit from locality.
To extend this idea even further, an LRU-approach with $N$ different 
buffers would lead to even fewer reads/writes from/to the SD-card.

\subsection{Limitations}
Due to time-constraints, no support for timestamps or read-only files is implemented.
Also, no estimates of free clusters are used. As previously state, we also didn't specify if a directory entry in a cluster is the last 
overall directory entry. Read-only flags are ignored and no support was added for designating files as read-only.
We also only use the first of the two File Allocation Tables.

\section{Filesystem Implementation}

\subsection{Requirements}
As said already, our filesystem is a filesystem as a service.
\begin{itemize}
    \item The filesystem is accessible for any user thread which has the \texttt{fs} library loaded
    \item The filesystem is responsible for thread-safety under the assumption that the underlying FAT32 implementation is not thread-safe
    \item It must keep track of any open file/directory in the whole system
    \item It should implement all the functions in \texttt{lib/fs/fopen.c} and \texttt{lib/fs/dirent.c} as to abstract the implementation from the use
\end{itemize}



\subsection{Approach}
For these requirements, the filesystem service is provided by a user process called \texttt{fs}.
This process handles all the requests to the filesystem using RPC. It is registered over the nameserver with the name \texttt{fs} and 
over that reacheable from any other user process. One subtlety is that the \texttt{fs} process itself can't use the filesystem over the same 
functions as the client (i.e. \texttt{fopen} won't work). The filesystem implements the RPC functions specified in the beginning of this chapter.

The other side of the implementation is the client side, which abstracts away the service using functions like \texttt{fopen, fclose}, etc.

To implement the client-side of the filesystem, we tried to keep the original approach of \texttt{ramfs}
as much as possible and replaced functions like \texttt{ramfs\_open}, \texttt{ramfs\_close}
by \texttt{aos\_rpc\_fs\_open}, \texttt{aos\_rpc\_fs\_close}, etc.

We store a opened file/directory in a similiar manner as in \texttt{ramfs}:

\begin{minted}{c}
struct fat32fs_dirent {
    char *name; 
    size_t size; 
    bool is_dir; 
    // cluster containing the directory-entry
    uint32_t dir_cluster; 
    // index into that cluster
    uint32_t dir_index; 
    // first data cluster of file
    uint32_t start_data_cluster; 
};

struct fat32fs_handle {
    int flags;
    // unique id of the opened file
    fileref_id_t fid;
    domainid_t pid;
    char *path;
    struct fat32fs_dirent *dirent;
    union {
        uint32_t dir_offset;
        uint32_t file_offset;
    } u;
    // current data cluster as 
    // specified by dir_offset/file_offset
    uint32_t curr_data_cluster;
};
\end{minted}
Of particular importance is the member \texttt{fid}.
This ID is a unique ID under all the opened files in the filesystem 
which is used to identify an opened handle in the \texttt{fs} process. Any RPC request to an opened
file identifies the handle uniquely by this FID. The FID is a global file descriptor, which is abstracted away in client processes.
Client requests to open file send this FID to talk to the service.

\subsubsection{Server Side}
The server side of the filesystem is implemented in \texttt{lib/fs/fat32fs.c} and builds on top of the FAT32 implementation.
In the \texttt{fs} process, the following happens when a new file/dir is opened:
\begin{itemize}
    \item Create new handle, store file-info into it
    \item Assign new unique FID to handle
    \item Add into hashmap \texttt{fid2handle}, which maps FID's to handles
    \item Append to hashmap \texttt{path2handle}, which maps a file path to a list of opened handles to that file
\end{itemize}
The first of the two hashmaps is used to find the handle of an opened file by FID in constant 
time. Using this second hash map, we resolve many thread-safety issues, which will be explained later.

On top of \texttt{fat32fs}, we register a nameservice called \texttt{fs} and with that a handler which responds to client requests.
This handler can be found in \texttt{lib/fs/fs\_rpc.c}.
Other than responding to requests, it has another important task: Thread Safety.
\subsection{Thread Safety}
To ensure two threads (or two processes) which access the same time can fulfill their requests, we implemented thread safety on top 
of the RPC handler in combination with the two hashmaps \texttt{path2handle, fid2handle}, which was mentioned previously.
Note first that most problems concerning thread safety are resolved already by locking on receival of a request and unlocking when
processing is done. 
Problems which are not resolved by that is anything that changes state in the handler itself. This includes for example:
\begin{itemize}
    \item Files can be requested to be deleted when they are opened somewhere else.
    \item When one processes increases the a file's size, this new size must be updated in the other open handles
    \item When one process decreases a file's size, the current file offsets of other opened handles to the same file might become invalid.
    \item 2 processes attempt to wrie to the same offset of the same file 
\end{itemize}
As a solution to those problems, we do the following.
\begin{itemize}
    \item Whenever a file is to be deleted but is already open, we deny the request.
    \item When a file size increases (due to new content written), we update all the other open handles to that file using the bookkeeping in \texttt{path2handle}
    \item Should a file size be decreased, we update any file offsets to other handles which are now out of bound to the end of the new file size.
    \item For writes to the same file, it is enough to handle them atomically by locking. Thus, both offsets will be written, but one of them 
    will overwrite the progress of the other one.
\end{itemize}

For processes which write to the same file offset on a same file, the result will be that both requests are fulfilled, however
after the first has been handled, the second will overwrite it. This has to be kept in mind by the programmer in userspace and is
not handled any further in the OS.

When a process changes the size of a file, we use the hashmap \texttt{path2fid} to update all handles which have the same file opened.
In addition, should a file size change, 

\subsection{Security}
For a full-blown OS, it makes sense to have a security layer in place which 
doesn't allow for one process to read/write using FID's of another process.
This could have been done in multiple ways. One could be to authenticate the RPC channels,
so that every FID is also linked to a specific channel. That way, requests to foreign 
FID's could be rejected. This could also have been done with special capabilities to
any opened file/directory. 

As this feature was of rather low priority given the mentioned problems we had during the integration, no additional security feature was implemented.

\section{ELF Loading}
Due to the problems during integration-phase, we didn't have time to implement
the ELF loading feature.

However, for the sake of completeness, we now state how this would be achieved.
Given our existing code, the approach we would follow is to use 
the file system to load the complete file into memory. After that,
any arguments to the program would be passed to by the shell.
To actually load the in-memory program, we can use existing functionality 
we have in our \texttt{spawn} library, where multiboot programs 
are also loaded into memory. After this step, we would start the program 
in the same way as we did for any other program.

\section{Performance Analysis}

\begin{figure}[h]
\centering
\includegraphics[scale=0.55]{fs_rw_new.png}
\caption{Read/Write Bandwidth with and without caching}
\label{fig:fs_rw_bw}
\end{figure}

In Figure~\ref{fig:fs_rw_bw}, we measured the read and write performance of our filesystem through a separate process which access the filesystem.
As you can see, the bandwidth stays very consistent in comparison with file sizes and we achieve a maximal read bandwidth of roughly 2.1KB/s,
which is very close to the block driver read bandwidth: 2.14KB/s. This can be explained by noting that in the sizes we have tested,
the FAT sector is cached all the time, so there is at most one read access for the FAT sector on which all of the read/write requests map to.
Thus, lookups to find next clusters in the linked list of clusters are near-instant, allowing the read-performance to be very close to the 
read-bandwidth of the block driver. If we look at the read-bandwidth of the filesystem without caching, it can be seen that now the FAT
sector needs to be looked up all the time, so performance drops there by roughly 250B/s, which is explained precisely by the additional read request into the 
FAT sector.

For writes, we are much lower as we now cannot make much use of the cached data, as write will always have to be performed. While the following reads
are still cached, we take a heavy toll because of those. Additionally, to keep the filesystem thread safe, after every RPC request for a write,
we update the directory entry with the new file size. Since the filesystem service is designed to write at most 1024 bytes per request, this 
furthermore gives use a 33\% overhead on the write bandwidth, as we have twice as many write requests.
Again, we see that caching still helps with the performance.


\section{Issues}
We already explained what features from the FAT32 specification are missing. Issues with the existing code are most of all
read performance, which is dominated by our slow RPC implementation. Continued work on this matter would be started there,
as it would resolve many performance problems.
\chapter{Nameservice}

This individual project is a central piece in pulling all the different components we have implemented
together into one system. With the nameserver components of the system providing a service, the servers,
can register themselves in the nameserver. This way their services become discoverable to components
which want to use them. They can now discover these by looking up the service name in the nameserver.
Using the nameserver they can then bind to the server and consume the services. The nameserver is highly
integrated with the RPC system as its main job apart from discovery is setting up communication channels
between processes.

\section{Interface \& Abstraction}
The interface to the nameserver enables to abstract away the underlying details of the RPC system. Such
details include which core a program is running on, consequently which RPC method to choose, how to send
and receive messages over it and how the proper receive handler is called.

The nameserver interface provides a register function where a server can provide its own message handler
and pointer to its internal state. With this message handler the server can establish its own messages
and communication protocols on top of the RPC system. Clients can then look up a registered service and
they receive an opaque channel over which they can send messages to the server and receive a response.
If servers also provide a library for message creation, the interaction between a client and a server
becomes fully abstracted from the underlying RPC system.

\section{The Nameserver}
\subsection{Names}
Before we get into the innards of the name server, we need to specify what a name is on our system.
We decided to use a hierarchical naming system where parts of the name are separated by a dot. The
parts of the name are snake case strings that may contain numbers. Further, a name part must start 
with a letter and end in a number or a letter. The full regular expressions to validate names is presented in
listing \ref{lst:name-regex}. Examples of such names are \sytx{net.icmp.ping_service} or \sytx{mem.server_v2}.
We provide functions to split names into their parts.

\begin{listing}
\begin{minted}{perl}
^[a-z][a-z0-9_]*[a-z0-9](\.[a-z][a-z0-9_]*[a-z0-9])*$
\end{minted}
\caption{Regular expression matching valid service names in our system}
\label{lst:name-regex}
\end{listing}

\subsection{Storing Names and Service Information}
Due to our choice of hierarchical names a tree was the obvious data structure to store names. Our name tree
consists of nodes that represent a part of a name. Each node has a pointer to a list of nodes which 
represent name parts that come later in a name than the part the current node refers to. So every level in
the node represents another level in the name hierarchy. Leaves that represent the actual names contain a
pointer to the information of the service with the name represented by the node and its location in the tree.
The tree has a root node which points to the name part list on the first level. This root node is never
deleted and does not itself carry a name part. It is simply the state pointer for the nameserver.

The service information contains all the information needed to identify and bind the service to the server providing. It contains the full name for convenience, the core the server runs on, the PID of the server
process, as well as the information on the handler function and the pointer to the handler state the server
supplies as it registers itself.

Looking up names is done by splitting up a name into parts and finding the proper node for a given part on
every level of the tree. If a service exists for a given name, then the node reached at the end of this
tree walk should contain a service information record. Note that this also allows for a service to be registered
at an intermediate node, i.e. there is a service with the name \sytx{foo.bar.baz} and another service
with the name \sytx{foo.bar}.

\subsection{Running the Server}
As a temporary (though not bad) solution to get the nameserver off the ground, it runs on the
init process on the bootstrap core. This is a convenient solution as it takes care of almost all bootstrapping
since all processes have a channel to their core local init process and all init processes on other cores
have a UMP channel to the bootstrap core. Communication to the nameserver just becomes a question of
relaying the calls to the proper core or calling init directly.

The only state of the nameserver is its name tree which is managed by the init process on the bootstrap core.
Running the server is simply init handling the incoming RPC calls and reading and writing the name tree.

\section{Registering a Service}
The registration of a service is straight forward. The server needs to provide a function pointer to its
receive handler and a pointer to its internal state if needed. With a call to the \mintinline{c}{nameservice_register}
function a new service information is created with all the information of the server, which is then sent
as a request to the nameserver with an RPC message. The nameserver then inserts the service information
into the name tree at the appropriate location given the service name.

\section{Binding to a Server}
If a client wants to bind to a server offering a service, it calls the \mintinline{c}{nameservice_lookup}
function. This function first looks up the service in the nameserver over RPC and gets the corresponding 
service information as a response if a service was registered under that name. For servers that take some
time to start up it is sometimes necessary to call the lookup function in a loop until the server is done
registering its service.

Determined by the core the server is running on, there are two ways to bind the server: over LMP or over UMP.
In both cases, a new direct RPC channel between the server and the client process is created. These bind
operations are centered around the init processes on the respective cores. In our system, the respective
init processes act as the monitor for their cores. Thus, they have all the information about the processes
running on their core such that they can forward bind requests to the server.

Both binding processes have a similar structure:
\begin{enumerate}
    \item The client creates a new endpoint (LMP endpoint or shared memory region for UMP) to set up its side
        of the channel.
    \item The client makes a bind request to establish a channel with a process by sending the server PID
        obtained from the service information and information about its side of the new channel.
    \item The init process on the core of the server process looks up the PID in its list of spawninfos
        and forwards the request over its LMP channel to the server.
    \item The server is able to set up its side of the channel fully based on the information it has received.
    \item The client receives information about the server's channel endpoint and is able to complete the
        setup of its side of the channel.
\end{enumerate}

The client (the caller of \sytx{nameservice_lookup} receives the opaque \sytx{nameservice_chan_t} binding
that contains the RPC binding, as well as the receive handler of the server and the pointer to the server state.

\subsection{LMP Binding}
As opposed to all the LMP channels that are created at spawn time of a process, the LMP binding operation
initializes a new channel in while the system is running. For LMP channels, this boils down to creating
new endpoints and exchanging the endpoint capabilities. 

Once the client knows it needs to create an LMP channel, it first creates a new LMP endpoint as its side of
the channel. This uses the same initialization procedure as at spawn time. The local endpoint capability and
the PID of the server are then sent to init as an LMP bind request. As stated above init forwards this message
to the server.

The server can fully initialize the channel using the received capability from the client as the channel's
remote capability. This channel is registered on the default eventset with a message handler that handles
handshakes and client requests. With this in place, the server initializes a handshake with the client in order to transfer
its own endpoint capability. The client is at this point already polling its incomplete channel for this message,
as it is eagerly awaiting the remote capability from the server to complete its side of the channel.

\subsection{UMP Binding}
As we encountered more and more bugs while integrating the nameserver with LMP binding together with the
other individual projects, we decided to deprioritize the UMP binding functionality in favor of the rest
of the system working. Unfortunately, every time work on this component started again, another bug blocking
progress for the team popped up such that the deprioritization ended up as a sacrifice of this functionality
for the sake of not introducing more bugs.

\section{Using the Connection}
Using the binding to the server is as simple as invoking \sytx{nameservice_rpc} with the nameservice channel
received from the lookup and the proper arguments. The library implements the sending between the client and
the server. Concretely, the client sends a request containing the message, in case of LMP the capability
(capability transfers over UMP are not supported as our group has not completed the \enquote{Capabilities Revisited}
milestone), as well as the receive handler function pointer and the pointer to the server state.

On the server side, the received message is unmarshalled and the received function pointer to the receive handler
is called with the proper arguments from the receive message and the pointers for the contents of the return message.
These responses from the handler function are then sent over the channel to the client.

\section{Challenges, Limitations, \& Improvements}
The most challenging part of implementing the nameserver were latent bugs or bad design decisions in the RPC system.
Implementing the LMP binding also lead to a significant rewrite of the LMP system (which would have been even
more significant given more time until the deadline). A common source of bugs was the combination of the
footguns that the C memory management provides and our twitchy trigger fingers. A particular favorite that comes to
mind is a buffer overrun that overwrote the function pointer of the handler function in our LMP channels
leading to heap addresses in the program counter. At this point, I want to acknowledge the good work of some
voluntary and involuntary rubber ducks that helped significantly in discovering such bugs.

A significant limitation is of course the missing functionality of UMP binding and thus also the ability
to connect and communicate to servers on a different core. For the same reason the service enumeration is not implemented.

Also on the topic of memory management is the distinct lack of cleanup of the channels between clients and
servers. Currently, these channels are allocated without any bookkeeping in place to be able to free them at
some point. Thus, these channels are currently lost to the memory nirvana. Given some more time, however, it
would be possible to add some state in processes keeping track of channels to servers or clients such that 
they can be properly cleaned up once they are not needed anymore.


\chapter{Networking}

\section{Access the hardware device registers}

The network process runs as a standalone process in user space. Therefore, it
needs to have access to the Device Frame capability that grants accesses to the
hardware device register. This was the starting point for this individual
project. We decided to pass the Device Frame in an arg cnode slot as it is
proposed in the manual. For this to work, we also needed to change the interface
of our spawn.h as currently we only could set up and dispatch the whole process
in once. But by having to provide a capability in a cnode slot we had to cut
those to parts into two, such that we could set up the process first, include
the capability in the arg cnode slot and only then dispatch the process.  The
corresponding part in the network process was significantly easier. We only
needed to map the capability in the arg cnode slot in to our virtual address
space.  Once, that was done, we could properly set up the network driver and see
the first packets arriving at our interface.


\section{Safe Send Queue}

For sending and receiving packets, the network driver has to communicate with
the hardware device.  The network driver works with two queues: the receiver and
the sender queue. If we want to send a packet, we have to warp it in a buffer
and then provide it to the sender queue. Those buffers cannot be at any
arbitrary place in the memory, but have to be in a region that has  been
registered beforehand with the queue.  As stated in the manual, it is important
to not enqueue two buffers that are stored at the same location in the region
twice before dequeueing. This is because,  every buffer has an owner, either the
network driver or the hardware device, and would result in undefined behaviour.
With this in mind, we developed a safe queue that provides an interface to place
some data in the sender queue. The safe queue simply keeps track of free buffers
in the region registered with the underlying sender queue. That means when we
want to enqueue a packet in the sender queue, we pass the packet to the safe
queue, and it will do some sanity checks, get a free buffer, copy the packet
into the free buffer and then enqueue the buffer into the sender queue, marking
the buffer as used.  There is no dequeue interface for the safe queue, but
somehow the buffers, that have been processed by the hardware device and are now
again owned by the network driver, have to be dequeued and marked as free
buffers again. This is done on every packet that is sent. While getting a free
buffer for the sending packet, the safe queue also dequeues all possible buffers
from the sender queue, such that we don't run out of free buffers.

The receiver queue, on the other hand, is much easier to handle. Here, we simply
dequeue a buffer that contains an incoming packet, extract the packet, pass the
packet to the packet handler of our network stack and then enqueue the buffer
again.

\section{General Receive Flow}

Upon receiving a packet, the packet is passed to a single point of entry, namely
the packet handler of our network stack.  The packet handler parses the packet,
determines the type of the packet, and passes it to more specialized handlers.
We implemented the handling of ARP packets and IP packets and with that
corresponding handlers.  While the ARP packet is already completely unwrapped,
the IP packet has to be parsed further, the type has to be determined again and
then has to be passed to another set of handlers. In this case, it may be an
ICMP, IGMP, UDP, UDPLITE or TCP packet type. We only implemented handlers for
ICMP and UDP packets.  Once a packet is completely unwrapped, it may be
processed correspondingly. That usually includes doing some legitimation checks,
storing the contained data, and may even trigger an immediate response. We
discuss this further in the upcoming sections.

\section{General Send Flow}

When we want to send data to another host, we first need to wrap that data into
a network packet. That may include several layers, depending on the type of data
we want to send. For this purpose, we implemented a packet assembler that is
capable of doing exactly that. Our packet assembler support assembling ARP, ICMP
and UDP packets. The assembled packet can then be enqueued in our safe sender
queue, ready for transmission.  Creating a convenient implementation for
assembling packets was a bit challenging. While the layering aspect of a network
stack is well known and also implemented in our packet handlers, it is a bit
harder to implement for packets assemblers than we initially thought. For
example, the final UDP packet has be in a continuous region of memory, starting
with the Ethernet header, then the IP header, then the UDP header and finally
the UDP payload.  If we want to follow the layering approach, then we would have
to first create the UDP packet, wrap it in a IP packet and finally in an
Ethernet packet. But as we start logically with and UDP packet, the memory
region starts with an Ethernet packet. So, either we have to allocate space for
the whole network packet, when we allocate the UDP packet or only allocate space
for the UDP packet, pass is to the next lower layer, where we allocate space for
both IP and UDP packet and copy the UDP packet from its previous memory region.
We even checked out the Linux network stack implementation for inspiration.  In
the end we chose our own, at the moment, convenient approach.  We provide packet
assemblers for ARP, ICMP and UDP packets, and each returns the final, completely
assembled packet. For example, the UDP packet assembler allocates space for the
whole packet including the Ethernet header, the IP header, the UDP header and
the UDP payload. It then starts filling the packet with data from the start of
the packet: first the Ethernet header, then the IP header, then the UDP header
and finally, its payload. This may not be the most scalable approach and may
introduce complication, when one wants to extend the network stack with new
layers or similar, but was at the moment of implementation the simplest way to
get started, so we kept this style for the all packet assemblers to be
consistent.

\section{ARP}

The address resolution protocol (ARP) is necessary to resolve a given IP to its
MAC address. This mapping is then stored in the ARP hash table for future use.

There are two possible types of ARP packets that our network stack may receive:
ARP requests and ARP responses. 

ARP requests arrive when anyone in the network wants to know the MAC address to
an IP address. So if the IP address is not our, we simply ignore the packets, as
we should not answer that request. On the other hand, if it is our IP address,
then we have to assemble an ARP packet that contains our MAC address and send a
ARP response to the original sender. At this point, we also store the IP to MAC
mapping in our ARP table such that we don't have to resolve this IP in the
future.

But if we don't know the mapping, we have to ask for it. Here, we assemble an
ARP request packet and broadcast it on the network, so that the owner of the IP
address may respond. When we receive it, we store the mapping in our ARP table.
At this point, we have to deal with the asynchronicity of the network. We need
the MAC address to continue, but also don't know when we will get the response
from the owner of the IP address.  Here, we tried two different approaches to
resolve an IP to a MAC address if we don't know the MAC address already. One was
to send the request and then wait for a given amount of time for the response
before returning an error. The other was to send the request and then
immediately return an error, implying that the caller has to handle that case
and to retry the resolution. We implemented the former approach.

\section{Ethernet}

This layer is relatively easy to implement. When we want to wrap an ARP or IP
packet into an Ethernet packet, we only have to prepend an Ethernet header
containing the source and destination MAC address and the type of the contained
data, i.e. ARP or IP.  Reception is also easy, as we only need the Ethernet type
to determine whether we have to call the ARP or IP packet handler.

\section{IP}

Assembling an IP header is also pretty straight forward with our assumptions we
have taken. Those include, we only assemble IPv4 packets, set the TTL to 128
only send IP packets that should not be fragmented.  Those assumptions also
simplify the handling of an incoming IP packet. We only process IPv4 packets
with a valid checksum and at least have the size of an IP header, and drop
fragmented IP packets. Based on the protocol, we then call the ICMP or UDP
handler, that we have implemented. All other protocols are not implemented, and
those packets are dropped.

\section{ICMP}

To successfully ping the Toradex board from another host, we have to reply to
valid ICMP echo requests with ICMP echo replies. An ICMP packet contains an
Ethernet header, an IP header, an ICMP header and some payload. With creating
the Ethernet and IP header as described above, the only part missing is the ICMP
header. Creating the ICMP header includes setting the ICMP type to echo replies,
the code to 0, the ID, sequence number and payload to the values that were
provided in the received ICMP echo packet and finally computing the check over
the ICMP header and payload. With that, we were able to ping our Toradex board
from our laptop!

On the other hand, we also wanted to ping other hosts from or Toradex board. So,
we have to be able to send ICMP echo requests and with that, we had to figure
out the meaning of the ID, sequence number and payload parameters.  As we know,
the ICMP serves the purpose of transmitting information and errors about the
IPv4. We are initially quite puzzled how an application, in this case "ping", is
able to receive incoming ICMP packets, as ICMP does not know the concept of
ports, therefore the well known demultiplexing process from UDP and TCP is not
possible here.  It turns out, that if one runs multiple "ping" instances on a
host, all "ping" instance get all ICMP responses and have to figure out which
one are responses to own requests. This is done with the ID parameter. We
decided to set the ID to our process number to distinguish the responses. We
also inspected how the real "ping" (i.e. the own installed on my laptop by my
package manager) chooses its ID and interestingly, it's not the process ID but
the value of a counter that increments with every "ping" instance. So, somehow,
every "ping" instance knows how many instances it is.  If you ever run the
"ping" command, you know that per default "ping" periodically send ICMP echo
packets to the provided host and reports the latency for each one until it is
terminated. As responses may arrive out of order, "ping" somehow needs to map
the response to the original request, and that is what the sequence number in
the ICMP header is used for.  Finally, ICMP packets may contain a payload and
under the hood "ping" puts random data into an ICMP echo packet and check if the
corresponding ICMP echo reply packet contains the same data and reports an error
otherwise.  This was an interesting part of building the network stack, as
"ping" is such a natural tool to use, one does not really think about its
internals.

With ICMP not knowing the concept of ports, we were also puzzled how we should
actually provide a ping instance with incoming ICMP packets. As we see in the
next section about UDP, every app has its socket with its own queue of incoming
packets. In the case of UDP, it is clear that every process has its own socket
and therefore it is straight forward on how the demultiplexing happens. In the
case of ICMP, it was not that clear. First, we had exactly one socket with one
queue that holds all incoming ICMP packets. But if we have two processes
consuming from the same socket, then packets intended for one process may be
consumed by the other, that simply ignores it and vice versa. So, we introduced
a queue for each process that want to consume ICMP packets and hence every
process gets every ICMP packet and has to do the demultiplexing in the process
itself, as it is intended. This of course brings a lot of packet duplication
with it. In the end we had also ICMP sockets like UDP socket, but here, they are
not distinguished by port, but by the process ID. 

\section{UDP}

We approached the implementation of UDP by setting up a UDP echo server directly
in the network stack. Therefore, if a UDP packet arrives at the UDP handler, we
immediately assembled a UDP packet and sent it back to the sender contains the
same payload. This allowed us to focus on assembling a UDP packet and setting up
the UDP header appropriately. Namely, a UDP packet contains an Ethernet header,
an IP header, the UDP header and the UDP payload. The UDP header contains the
source and destination port and the total length of the header and payload.  But
we had a major drawback here: we could not make the checksum work. Somehow, the
bytes representing the checksum in a UDP header were different when parsed on
the board and when inspected in Wireshark in the same packet. All other bytes
around the checksum matched, i.e. source and destination port, length, payload,
but not those representing the checksum. We had no idea why this would happen
and as UDP checksum is not mandatory in IPv4, we simply skipped that part for
both sending (creating the checksum) and receiving (verifying the checksum).
Once, we were confident with implementing the UDP correctly, we started
implementing UDP sockets. This is needed to demultiplex incoming UDP packets
based on their destination port to the corresponding client application that is
listening on that port.  A UDP socket is simply a producer/consumer queue. When
a UDP packet arrive in our network stack, the right socket is found based on the
port and its payload along with some needed meta information such as the source
IP and port is placed at the producer size of the queue. The client application
is then responsible to repeatedly poll this queue for new packets to consume.
This polling functionality is provided by the network service, describe in the
section below.  Other functionalities provided to the client application are of
course first creating a socket and also sending over the socket. It is now
allowed to have two UDP sockets listening on the same port, and therefore this
is checked in the socket creation. All existing UDP sockets form a simply linked
list and therefore new UDP socket are appended to this linked list. Sending over
a socket is putting already known parts together, such as getting the MAC for an
IP, assembling a UDP packet and putting it into our safe sender queue.

\section{Debugging}

The main tool in use to debug our network stack was Wireshark. It gave us
confident that the packets, our network stack is sending, are correct, as
Wireshark was able to parse it. Also, the other way around it helped a lot,
knowing what an incoming packet contains so, we could set up our handlers
correctly.

Other tools we used to trigger the correct path in our network stack were
"arping" to send a ARP request to the Toradex board, "arp" to check the ARP
table on the host to ensure that our network stack provided the correct MAC
address, "ping" for ICMP echo packets, "nc" both as a UDP client and UDP server
to test the sending and receiving functionality of UDP packets in our network
stack.

Further, we implemented debug print functions for every packet type (Ethernet,
IP, ARP, ICMP and UDP), the ARP table, IP and MAC addresses and also the
producer/consumer queue in our sockets to have some insights of what is
happening in our network stack.

\section{UDP Hack}

While implementing the UDP part in the network stack, we had to apply some hacks
to test the functionality. One was already mentioned above, by echo UDP packets
directly in the network stack.  Another one, to test the reception of UDP
packets from the client application's perspective, that means consuming the
packets that are stored in the producer/consumer queue in a UDP socket, we
proceeded as follows. Upon reception of a UDP packet in the network stack, we
stored the UDP packet as described above in a UDP socket, but then immediately
consumed it over the same interface that would a client application use to
receive a UDP packet and also send some payload back over the interface. This
required of course to create a UDP socket statically at the setup of the network
process, to make the UDP handler not drop the packet.

\section{ICMP Hack}

The UDP hacks described above are quite straight forward. Testing the ICMP part
is a bit more involved. Responding to ICMP echo requests required no hacks, as
it can be directly testing by using "ping" on another host.  But issuing own
ICMP echo requests is trickier. At this point, we have not yet implemented our
own "ping" and also do not have our nameserver setup yet. So communication
between any two processes is not yet implemented, hence we had to apply another
hack to make this work directly in our network stack: Upon handling a ICMP echo
request, we do not directly return, but send our own ICMP echo request to the
host, that just pinged us with some dummy ID, sequence number and no payload and
then try to consume an ICMP echo reply over the ICMP socket like a client
application. On the first try, consuming the ICMP echo reply will fail, as the
request has not yet made the whole roundtrip to the host and back. When we send
another ICMP echo request from the host to the Toradex board, we are going to
again handle it, issue our own ICMP echo request, but at this consuming from the
ICMP socket will not fail, because the ICMP echo reply from our previous ICMP
echo request, will be stored in there. This way, we can then check if the ID,
sequence number and payload align with our dummy values.

\section{Network Service}

Such that other processes, like multiple UDP servers, can use the network stack,
we need to expose our network stack as a network service over the nameserver
that another team member has implemented.  For this, we register the network
service with a service name and a handler at the nameserver. On the other side,
we implemented an interface in the aos library that allow other processes to
communicate with the network stack. The main purpose of the mentioned handler
and the interface is to marshal and unmarshal message that are transmitted over
RPC.  Supported operations from a client application's perspective are to create
and destroy a socket and send and receive over a socket.

\section{Benchmarks}

Finally, we did some extensive benchmarking of the network stack. For this, we
used the "get\_system\_time()" function provided by "aos/deferred.h" to measure
the ticks used to execute a specific part of the code. We introduced
benchmarking levels, such that we were able to have provided drill down results
without having to deal with overhead of nested benchmarking.  The benchmarking
results are printed to stdout and collected from there to be analysed by a
script.

% ICMP
\begin{table}
    \begin{tabular}{|llllll|l|l|l|l|l|l|}
    \hline
    \multicolumn{6}{|l|}{step} & mean & median & std & min & max & n \\ \hline
    \multicolumn{1}{|l|}{} & \multicolumn{5}{l|}{handle packet} & 69976 & 61746 & 26291 & 59876 & 190546 & 60 \\
    \multicolumn{1}{|l|}{} & \multicolumn{1}{l|}{} & \multicolumn{4}{l|}{handle ip packet} & 68366 & 61851 & 22964 & 60088 & 191211 & 60 \\
    \multicolumn{1}{|l|}{} & \multicolumn{1}{l|}{} & \multicolumn{1}{l|}{} & \multicolumn{3}{l|}{handle icmp packet} & 68827 & 62220 & 23229 & 60018 & 193761 & 60 \\
    \multicolumn{1}{|l|}{} & \multicolumn{1}{l|}{} & \multicolumn{1}{l|}{} & \multicolumn{1}{l|}{} & \multicolumn{2}{l|}{assemble icmp packet} & 33461 & 30656 & 24713 & 12518 & 142123 & 60 \\
    \multicolumn{1}{|l|}{} & \multicolumn{1}{l|}{} & \multicolumn{1}{l|}{} & \multicolumn{1}{l|}{} & \multicolumn{1}{l|}{} & malloc response packet & 32653 & 29924 & 23953 & 12302 & 138352 & 60 \\
    \multicolumn{1}{|l|}{} & \multicolumn{1}{l|}{} & \multicolumn{1}{l|}{} & \multicolumn{1}{l|}{} & \multicolumn{1}{l|}{} & create icmp packet & 13 & 14 & 1 & 10 & 17 & 60 \\
    \multicolumn{1}{|l|}{} & \multicolumn{1}{l|}{} & \multicolumn{1}{l|}{} & \multicolumn{1}{l|}{} & \multicolumn{1}{l|}{} & create ip packet & 17 & 17 & 1 & 15 & 20 & 60 \\
    \multicolumn{1}{|l|}{} & \multicolumn{1}{l|}{} & \multicolumn{1}{l|}{} & \multicolumn{1}{l|}{} & \multicolumn{2}{l|}{enqueue icmp packet} & 35516 & 48775 & 19563 & 12667 & 73014 & 60 \\
    \hline
    \end{tabular}
    \caption{ICMP benchmarks}
    \label{tab:icmp-benchmarks}
\end{table}

We benchmarked incoming ICMP echo requests with a ICMP echo reply response and
receiving UDP packets. All tests are done repeatedly to get meaningful results. 
The results are shown in table \ref{tab:icmp-benchmarks} and
\ref{tab:udp-benchmarks} respectively. At first, we thought that ticks in the
order of magnitude of several thousand is the standard. But as we further
benchmarked our code, we found that some code secions are executed in a matter
of a few ticks. This made as curious and we drilled down into the network stack.
In the end it turned out, that a single malloc takes about 30'000 ticks. That
means assembling a packet and placing the packet in the safe sender queue, each
yields a malloc and therefore a total sum of 60 thousand ticks have to accounted
for malloc calls for sending a single packet.  Hence, the best way to improve
the performance of our network stack would be to optimize the memory management.

We also tried to saturate the network link with packets. For that we sent about
16 packets, each carrying about one thousand bytes, per second to the Toradex
board resulting over the half of the packets being dropped.

Under this load the latency also increased up to 22 seconds on average.

Under normal load ICMP echo requests took about 16 miliseconds to be answered
(min: 7ms, avg: 16ms, max: 482ms, std: 60ms).

% UDP
\begin{table}
    \begin{tabular}{|llll|l|l|l|l|l|l|}
    \hline
    \multicolumn{4}{|l|}{step} & mean & median & std & min & max & n \\ \hline
    \multicolumn{4}{|l|}{handle packet} & 5385 & 8 & 26458 & 5 & 176355 & 68 \\
    \multicolumn{1}{|l|}{} & \multicolumn{3}{l|}{handle ip packet} & 8 & 7 & 4 & 3 & 21 & 64 \\
    \multicolumn{1}{|l|}{} & \multicolumn{1}{l|}{} & \multicolumn{2}{l|}{handle udp packet} & 3 & 3 & 2 & 1 & 17 & 63 \\
    \multicolumn{1}{|l|}{} & \multicolumn{1}{l|}{} & \multicolumn{1}{l|}{} & process udp packet & 5 & 3 & 5 & 1 & 18 & 63 \\
    \hline
    \end{tabular}
    \caption{UDP benchmarks}
    \label{tab:udp-benchmarks}
\end{table}

\chapter{Summary}

\section{Open Points}
Our Operating System has seen many ups and downs.
In this chapter we want to re-emphasize again what performance issues and bugs we experienced in our system.
We also want to address the most important design decisions we have been questioning and reconsidering in hindsight.

We had to deal with many performance issues, most notably:
\begin{itemize}
\item Slow RPC due to a UMP channel being polled in a empty while loop, waiting for new messages to arrive. We found out that 
RPC was slow because the thread which was polling inside of the while loop was scheduled too often, starving the rest of the system
of execution time. After using \texttt{thread\_yield} in this UMP while loop, things were tens of times faster.
\item Paging Unmap: While the implementation we do is much better than naively iterating through each page, it can be done much more efficiently.
This is a problem which persists up until now, as we couldn't allocate the necessary time to replace it with a better solution.
\item UMP channels currently run inside one thread each, as they need to be polled. We recently heard about polled waitsets, which could be
used to avoid huge overhead if there were many UMP channels (which in turn would have required many active threads in a process)
\item While we are quite happy with the speed of the filesystem, there could be huge gains in performance if we had a faster block driver.
Thinking about modern operating systems and the use of filesystems on such, it has been eye-opening how fast one can do things and how much more 
complicated they are than one might think.
\item One thing each process has plenty of are page faults. For most of our paging core, we have huge locks around complete functions.
While we are not completely sure, we think some finer grained locking might still be possible.
\end{itemize}

The next thing we also want to address is known bugs:
\begin{itemize}
\item  We believe our system contains quite some memory leaks at this point. Unfortunately, addressing these issues was not the highest priority of our system, noting that three days before the submission of this report, the system hasn't been working at all since we integrated our individual milestones together.
\item Our paging implementation is assumed to have a bug in it. We have experience multiple times that our OS would fail to start 
new processes over the shell, failing with an error saying we attempted to map a page table entry that is already mapped. We were close enough
to locate the problem to a single paging unmap call, without which the errors wouldn't re-appear as often. This call was in our spawn library and
it was supposed to unmap the memory containing a binary to spawn, which was mapped into the init process so that it can 
pass it into the child vspace. We decided to remove this paging unmap, leaving open a memory leak, but ensuring a much more stable overall system.
\item A  concrete demonstration to the last point is the binary spawnTester, which doesn't end successfully and runs into a 
"Vnode already mapped" error. The source of the place where a previous code execution got to map to something preexisting is unknown to us and our debugging
wasn't successful. However, it seems like there is some place in our code where our virtual memory abstraction was not implemented correctly. In spawnTester,
this happens after spawning 5 instances have been spawned. 
\end{itemize}

This list is of course by no means complete, however these are the bugs that have been our main focus up to now.

As a final point, we address some decisions which in hindsight were not good in our eyes.
\begin{itemize}
    \item UMP channels using a single thread each to poll for a message. This is not scalable at all and should have been solved in a 
    different way.
    \item RPC clients have an interface for sending and receiving messages atomically. The reception of a response from a server
    is done using polling instead of using upcalls, as it is done for reception of a client request on the server. While this idea
    made sense at the time, we now don't approve of this design decision at all.
\end{itemize}

\section{Final Word}
We have implemented a system with an enjoyable front-end shell, backed by a filesystem and networking abilities, which in turn 
again are backed by our nameserver. Besides these services we've implemented in the individual projects, a lot more has been created.
During the course, we implemented and learned what it means to communicate over processes, run and communicate over different cores,
how sophisticated bulk memory allocation for managing physical memory is. We learned what it means to start a new thread,
a new process and how to provide these processes with memory and all the resources it needs to be provide a suitable
environment for custom user processes.

It has been an incredible journey to build this system. Unlike other courses at ETH, we were not forced into a direction
by a bunch of skeleton code and instructions but instead were free to explode and be creative. Many design decisions had to
be made. If they were wrong, we had to deal with the consequences. Although it required a vast amount of work each week,
it has been incredibly rewarding. We think back to Professor Roscoe's first lecture in this course, where he said that 
building operating systems cannot be learned in a book. Now we know what he meant by that and we must fully agree.


\end{document}