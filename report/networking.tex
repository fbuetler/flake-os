\chapter{Networking}

\section{Access the hardware device registers}

The network process runs as a standalone process in user space. Therefore, it
needs to have access to the Device Frame capability that grants accesses to the
hardware device register. This was the starting point for this individual
project. We decided to pass the Device Frame in an arg cnode slot as it is
proposed in the manual. For this to work, we also needed to change the interface
of our spawn.h as currently we only could set up and dispatch the whole process
in once. But by having to provide a capability in a cnode slot we had to cut
those to parts into two, such that we could set up the process first, include
the capability in the arg cnode slot and only then dispatch the process.  The
corresponding part in the network process was significantly easier. We only
needed to map the capability in the arg cnode slot in to our virtual address
space.  Once, that was done, we could properly set up the network driver and see
the first packets arriving at our interface.


\section{Safe Send Queue}

For sending and receiving packets, the network driver has to communicate with
the hardware device.  The network driver works with two queues: the receiver and
the sender queue. If we want to send a packet, we have to warp it in a buffer
and then provide it to the sender queue. Those buffers cannot be at any
arbitrary place in the memory, but have to be in a region that has  been
registered beforehand with the queue.  As stated in the manual, it is important
to not enqueue two buffers that are stored at the same location in the region
twice before dequeueing. This is because,  every buffer has an owner, either the
network driver or the hardware device, and would result in undefined behaviour.
With this in mind, we developed a safe queue that provides an interface to place
some data in the sender queue. The safe queue simply keeps track of free buffers
in the region registered with the underlying sender queue. That means when we
want to enqueue a packet in the sender queue, we pass the packet to the safe
queue, and it will do some sanity checks, get a free buffer, copy the packet
into the free buffer and then enqueue the buffer into the sender queue, marking
the buffer as used.  There is no dequeue interface for the safe queue, but
somehow the buffers, that have been processed by the hardware device and are now
again owned by the network driver, have to be dequeued and marked as free
buffers again. This is done on every packet that is sent. While getting a free
buffer for the sending packet, the safe queue also dequeues all possible buffers
from the sender queue, such that we don't run out of free buffers.

The receiver queue, on the other hand, is much easier to handle. Here, we simply
dequeue a buffer that contains an incoming packet, extract the packet, pass the
packet to the packet handler of our network stack and then enqueue the buffer
again.

\section{General Receive Flow}

Upon receiving a packet, the packet is passed to a single point of entry, namely
the packet handler of our network stack.  The packet handler parses the packet,
determines the type of the packet, and passes it to more specialized handlers.
We implemented the handling of ARP packets and IP packets and with that
corresponding handlers.  While the ARP packet is already completely unwrapped,
the IP packet has to be parsed further, the type has to be determined again and
then has to be passed to another set of handlers. In this case, it may be an
ICMP, IGMP, UDP, UDPLITE or TCP packet type. We only implemented handlers for
ICMP and UDP packets.  Once a packet is completely unwrapped, it may be
processed correspondingly. That usually includes doing some legitimation checks,
storing the contained data, and may even trigger an immediate response. We
discuss this further in the upcoming sections.

\section{General Send Flow}

When we want to send data to another host, we first need to wrap that data into
a network packet. That may include several layers, depending on the type of data
we want to send. For this purpose, we implemented a packet assembler that is
capable of doing exactly that. Our packet assembler support assembling ARP, ICMP
and UDP packets. The assembled packet can then be enqueued in our safe sender
queue, ready for transmission.  Creating a convenient implementation for
assembling packets was a bit challenging. While the layering aspect of a network
stack is well known and also implemented in our packet handlers, it is a bit
harder to implement for packets assemblers than we initially thought. For
example, the final UDP packet has be in a continuous region of memory, starting
with the Ethernet header, then the IP header, then the UDP header and finally
the UDP payload.  If we want to follow the layering approach, then we would have
to first create the UDP packet, wrap it in a IP packet and finally in an
Ethernet packet. But as we start logically with and UDP packet, the memory
region starts with an Ethernet packet. So, either we have to allocate space for
the whole network packet, when we allocate the UDP packet or only allocate space
for the UDP packet, pass is to the next lower layer, where we allocate space for
both IP and UDP packet and copy the UDP packet from its previous memory region.
We even checked out the Linux network stack implementation for inspiration.  In
the end we chose our own, at the moment, convenient approach.  We provide packet
assemblers for ARP, ICMP and UDP packets, and each returns the final, completely
assembled packet. For example, the UDP packet assembler allocates space for the
whole packet including the Ethernet header, the IP header, the UDP header and
the UDP payload. It then starts filling the packet with data from the start of
the packet: first the Ethernet header, then the IP header, then the UDP header
and finally, its payload. This may not be the most scalable approach and may
introduce complication, when one wants to extend the network stack with new
layers or similar, but was at the moment of implementation the simplest way to
get started, so we kept this style for the all packet assemblers to be
consistent.

\section{ARP}

The address resolution protocol (ARP) is necessary to resolve a given IP to its
MAC address. This mapping is then stored in the ARP hash table for future use.

There are two possible types of ARP packets that our network stack may receive:
ARP requests and ARP responses. 

ARP requests arrive when anyone in the network wants to know the MAC address to
an IP address. So if the IP address is not our, we simply ignore the packets, as
we should not answer that request. On the other hand, if it is our IP address,
then we have to assemble an ARP packet that contains our MAC address and send a
ARP response to the original sender. At this point, we also store the IP to MAC
mapping in our ARP table such that we don't have to resolve this IP in the
future.

But if we don't know the mapping, we have to ask for it. Here, we assemble an
ARP request packet and broadcast it on the network, so that the owner of the IP
address may respond. When we receive it, we store the mapping in our ARP table.
At this point, we have to deal with the asynchronicity of the network. We need
the MAC address to continue, but also don't know when we will get the response
from the owner of the IP address.  Here, we tried two different approaches to
resolve an IP to a MAC address if we don't know the MAC address already. One was
to send the request and then wait for a given amount of time for the response
before returning an error. The other was to send the request and then
immediately return an error, implying that the caller has to handle that case
and to retry the resolution. 

We implemented the later approach, as % TODO that may work now

\section{Ethernet}

This layer is relatively easy to implement. When we want to wrap an ARP or IP
packet into an Ethernet packet, we only have to prepend an Ethernet header
containing the source and destination MAC address and the type of the contained
data, i.e. ARP or IP.  Reception is also easy, as we only need the Ethernet type
to determine whether we have to call the ARP or IP packet handler.

\section{IP}

Assembling an IP header is also pretty straight forward with our assumptions we
have taken. Those include, we only assemble IPv4 packets, set the TTL to 128
only send IP packets that should not be fragmented.  Those assumptions also
simplify the handling of an incoming IP packet. We only process IPv4 packets
with a valid checksum and at least have the size of an IP header, and drop
fragmented IP packets. Based on the protocol, we then call the ICMP or UDP
handler, that we have implemented. All other protocols are not implemented, and
those packets are dropped.

\section{ICMP}

To successfully ping the Toradex board from another host, we have to reply to
valid ICMP echo requests with ICMP echo replies. An ICMP packet contains an
Ethernet header, an IP header, an ICMP header and some payload. With creating
the Ethernet and IP header as described above, the only part missing is the ICMP
header. Creating the ICMP header includes setting the ICMP type to echo replies,
the code to 0, the ID, sequence number and payload to the values that were
provided in the received ICMP echo packet and finally computing the check over
the ICMP header and payload. With that, we were able to ping our Toradex board
from our laptop!

On the other hand, we also wanted to ping other hosts from or Toradex board. So,
we have to be able to send ICMP echo requests and with that, we had to figure
out the meaning of the ID, sequence number and payload parameters.  As we know,
the ICMP serves the purpose of transmitting information and errors about the
IPv4. We are initially quite puzzled how an application, in this case "ping", is
able to receive incoming ICMP packets, as ICMP does not know the concept of
ports, therefore the well known demultiplexing process from UDP and TCP is not
possible here.  It turns out, that if one runs multiple "ping" instances on a
host, all "ping" instance get all ICMP responses and have to figure out which
one are responses to own requests. This is done with the ID parameter. We
decided to set the ID to our process number to distinguish the responses. We
also inspected how the real "ping" (i.e. the own installed on my laptop by my
package manager) chooses its ID and interestingly, it's not the process ID but
the value of a counter that increments with every "ping" instance. So, somehow,
every "ping" instance knows how many instances it is.  If you every run the
"ping" command, you know that per default "ping" periodically send ICMP echo
packets to the provided host and reports the latency for each one until it is
terminated. As responses may arrive out of order, "ping" somehow needs to map
the response to the original request, and that is what the sequence number in
the ICMP header is used for.  Finally, ICMP packets may contain a payload and
under the hood "ping" puts random data into an ICMP echo packet and check if the
corresponding ICMP echo reply packet contains the same data and reports an error
otherwise.  This was an interesting part of building the network stack, as
"ping" is such a natural tool to use, one does not really think about its
internals.

% TODO
With ICMP not knowing the concept of ports, we were also puzzled how we should
actually provide a ping instance with incoming ICMP packets. As we see in the
next section about UDP, every app has its socket with  ...

% Talk about the bug: exactly one socket

\section{UDP}

We approached the implementation of UDP by setting up a UDP echo server directly
in the network stack. Therefore, if a UDP packet arrives at the UDP handler, we
immediately assembled a UDP packet and sent it back to the sender contains the
same payload. This allowed us to focus on assembling a UDP packet and setting up
the UDP header appropriately. Namely, a UDP packet contains an Ethernet header,
an IP header, the UDP header and the UDP payload. The UDP header contains the
source and destination port and the total length of the header and payload.  But
we had a major drawback here: we could not make the checksum work. Somehow, the
bytes representing the checksum in a UDP header were different when parsed on
the board and when inspected in Wireshark in the same packet. All other bytes
around the checksum matched, i.e. source and destination port, length, payload,
but not those representing the checksum. We had no idea why this would happen
and as UDP checksum is not mandatory in IPv4, we simply skipped that part for
both sending (creating the checksum) and receiving (verifying the checksum).
Once, we were confident with implementing the UDP correctly, we started
implementing UDP sockets. This is needed to demultiplex incoming UDP packets
based on their destination port to the corresponding client application that is
listening on that port.  A UDP socket is simply a producer/consumer queue. When
a UDP packet arrive in our network stack, the right socket is found based on the
port and its payload along with some needed meta information such as the source
IP and port is placed at the producer size of the queue. The client application
is then responsible to repeatedly poll this queue for new packets to consume.
This polling functionality is provided by the network service, describe in the
section below.  Other functionalities provided to the client application are of
course first creating a socket and also sending over the socket. It is now
allowed to have two UDP sockets listening on the same port, and therefore this
is checked in the socket creation. All existing UDP sockets form a simply linked
list and therefore new UDP socket are appended to this linked list. Sending over
a socket is putting already known parts together, such as getting the MAC for an
IP, assembling a UDP packet and putting it into our safe sender queue.

\section{Debugging}

The main tool in use to debug our network stack was Wireshark. It gave us
confident that the packets, our network stack is sending, are correct, as
Wireshark was able to parse it. Also, the other way around it helped a lot,
knowing what an incoming packet contains so, we could set up our handlers
correctly.

Other tools we used to trigger the correct path in our network stack were
"arping" to send a ARP request to the Toradex board, "arp" to check the ARP
table on the host to ensure that our network stack provided the correct MAC
address, "ping" for ICMP echo packets, "nc" both as a UDP client and UDP server
to test the sending and receiving functionality of UDP packets in our network
stack.

Further, we implemented debug print functions for every packet type (Ethernet,
IP, ARP, ICMP and UDP), the ARP table, IP and MAC addresses and also the
producer/consumer queue in our sockets to have some insights of what is
happening in our network stack.

\section{UDP Hack}

While implementing the UDP part in the network stack, we had to apply some hacks
to test the functionality. One was already mentioned above, by echo UDP packets
directly in the network stack.  Another one, to test the reception of UDP
packets from the client application's perspective, that means consuming the
packets that are stored in the producer/consumer queue in a UDP socket, we
proceeded as follows. Upon reception of a UDP packet in the network stack, we
stored the UDP packet as described above in a UDP socket, but then immediately
consumed it over the same interface that would a client application use to
receive a UDP packet and also send some payload back over the interface. This
required of course to create a UDP socket statically at the setup of the network
process, to make the UDP handler not drop the packet.

\section{ICMP Hack}

The UDP hacks described above are quite straight forward. Testing the ICMP part
is a bit more involved. Responding to ICMP echo requests required no hacks, as
it can be directly testing by using "ping" on another host.  But issuing own
ICMP echo requests is trickier. At this point, we have not yet implemented our
own "ping" and also do not have our nameserver setup yet. So communication
between any two processes is not yet implemented, hence we had to apply another
hack to make this work directly in our network stack: Upon handling a ICMP echo
request, we do not directly return, but send our own ICMP echo request to the
host, that just pinged us with some dummy ID, sequence number and no payload and
then try to consume an ICMP echo reply over the ICMP socket like a client
application. On the first try, consuming the ICMP echo reply will fail, as the
request has not yet made the whole roundtrip to the host and back. When we send
another ICMP echo request from the host to the Toradex board, we are going to
again handle it, issue our own ICMP echo request, but at this consuming from the
ICMP socket will not fail, because the ICMP echo reply from our previous ICMP
echo request, will be stored in there. This way, we can then check if the ID,
sequence number and payload align with our dummy values.

\section{Network Service}

Such that other processes, like multiple UDP servers, can use the network stack,
we need to expose our network stack as a network service over the nameserver
that another team member has implemented.  For this, we register the network
service with a service name and a handler at the nameserver. On the other side,
we implemented an interface in the aos library that allow other processes to
communicate with the network stack. The main purpose of the mentioned handler
and the interface is to marshal and unmarshal message that are transmitted over
RPC.  Supported operations from a client application's perspective are to create
and destroy a socket and send and receive over a socket.

\section{Benchmarks}

Finally, we did some extensive benchmarking of the network stack. For this, we
used the "get\_system\_time()" function provided by "aos/deferred.h" to measure
the ticks used to execute a specific part of the code. We introduced
benchmarking levels, such that we were able to have provided drill down results
without having to deal with overhead of nested benchmarking.  The benchmarking
results are printed to stdout and collected from there to be analysed by a
script.

% ICMP
\begin{table}[]
    \begin{tabular}{|llllll|l|l|l|l|l|l|}
    \hline
    \multicolumn{6}{|l|}{step} & mean & median & std & min & max & n \\ \hline
    \multicolumn{1}{|l|}{} & \multicolumn{5}{l|}{handle packet} & 69976 & 61746 & 26291 & 59876 & 190546 & 60 \\
    \multicolumn{1}{|l|}{} & \multicolumn{1}{l|}{} & \multicolumn{4}{l|}{handle ip packet} & 68366 & 61851 & 22964 & 60088 & 191211 & 60 \\
    \multicolumn{1}{|l|}{} & \multicolumn{1}{l|}{} & \multicolumn{1}{l|}{} & \multicolumn{3}{l|}{handle icmp packet} & 68827 & 62220 & 23229 & 60018 & 193761 & 60 \\
    \multicolumn{1}{|l|}{} & \multicolumn{1}{l|}{} & \multicolumn{1}{l|}{} & \multicolumn{1}{l|}{} & \multicolumn{2}{l|}{assemble icmp packet} & 33461 & 30656 & 24713 & 12518 & 142123 & 60 \\
    \multicolumn{1}{|l|}{} & \multicolumn{1}{l|}{} & \multicolumn{1}{l|}{} & \multicolumn{1}{l|}{} & \multicolumn{1}{l|}{} & malloc response packet & 32653 & 29924 & 23953 & 12302 & 138352 & 60 \\
    \multicolumn{1}{|l|}{} & \multicolumn{1}{l|}{} & \multicolumn{1}{l|}{} & \multicolumn{1}{l|}{} & \multicolumn{1}{l|}{} & create icmp packet & 13 & 14 & 1 & 10 & 17 & 60 \\
    \multicolumn{1}{|l|}{} & \multicolumn{1}{l|}{} & \multicolumn{1}{l|}{} & \multicolumn{1}{l|}{} & \multicolumn{1}{l|}{} & create ip packet & 17 & 17 & 1 & 15 & 20 & 60 \\
    \multicolumn{1}{|l|}{} & \multicolumn{1}{l|}{} & \multicolumn{1}{l|}{} & \multicolumn{1}{l|}{} & \multicolumn{2}{l|}{enqueue icmp packet} & 35516 & 48775 & 19563 & 12667 & 73014 & 60 \\
    \hline
    \end{tabular}
    \caption{ICMP benchmarks}
    \label{tab:icmp-benchmarks}
\end{table}

We benchmarked incoming ICMP echo requests with a ICMP echo reply response and
receiving UDP packets. All tests are done repeatedly to get meaningful results. 
The results are shown in table \ref{tab:icmp-benchmarks} and
\ref{tab:udp-benchmarks} respectively. At first, we thought that ticks in the
order of magnitude of several thousand is the standart. But as we further
benchmarked our code, we found that some code secions are executed in a matter
of a few ticks. This made as curious and we drilled down into the network stack.
In the end it turned out, that a single mallpc takes aboute 30'000 ticks. That
means assembling a packet and placing the packet in the safe sender queue, each
yields a malloc and therefore a total sum om 60 thousand ticks have to accounted
for malloc calls for sending a single packet.  Hence, the best way to improve
the perfmance of our network stack would be to optimize the memory management.

We also tried to saturate the network link with packets. For that we sent about
16 packets, each carrying about one thousand bytes, per second to the Toradex
board resulting over the half of the packets being dropped.

Under this load the latency also increased up to 22 seconds on average.

Under normal load ICMP echo requests took about 16 miliseconds to be answered
(min: 7ms, avg: 16ms, max: 482ms, std: 60ms).

% UDP
\begin{table}[]
    \begin{tabular}{|llll|l|l|l|l|l|l|}
    \hline
    \multicolumn{4}{|l|}{step} & mean & median & std & min & max & n \\ \hline
    \multicolumn{4}{|l|}{handle packet} & 5385 & 8 & 26458 & 5 & 176355 & 68 \\
    \multicolumn{1}{|l|}{} & \multicolumn{3}{l|}{handle ip packet} & 8 & 7 & 4 & 3 & 21 & 64 \\
    \multicolumn{1}{|l|}{} & \multicolumn{1}{l|}{} & \multicolumn{2}{l|}{handle udp packet} & 3 & 3 & 2 & 1 & 17 & 63 \\
    \multicolumn{1}{|l|}{} & \multicolumn{1}{l|}{} & \multicolumn{1}{l|}{} & process udp packet & 5 & 3 & 5 & 1 & 18 & 63 \\
    \hline
    \end{tabular}
    \caption{UDP benchmarks}
    \label{tab:udp-benchmarks}
\end{table}
